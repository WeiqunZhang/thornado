\section{Realizability-Preserving Limiter}
\label{sec:limiter}

The bound-preserving DG-IMEX method developed in previous sections is designed to preserve realizability of the cell averaged moments, i.e., $\vect{\cM}_{\bK}\in\cR$, provided sufficiently accurate quadratures are used to integrate integrals in the DG method, specific CFL conditions are satisfied, and that the polynomial approximation $\vect{\cM}_{h}$, at time $t^{n}$, is realizable in a set of quadrature points in each element $\bK$.  
We denote this quadrature set by $S\subset\bK$.  
In the DG method, we use the limiter proposed by Zhang \& Shu \cite{zhangShu_2010a} for scalar conservation laws to enforce the bounds on the zeroth moment $\cJ$ (see also \cite{liuOsher_1996}).  
We replace the polynomial $\cJ_{h}^{n}(\vect{z})$ with the limited polynomial
\begin{equation}
  \tilde{\cJ}_{h}^{n}(\vect{z})
  =\vartheta_{1}\,\cJ_{h}^{n}(\vect{z})+(1-\vartheta_{1})\,\cJ_{\bK}^{n},
  \label{eq:limitDensity}
\end{equation}
where the limiter parameter $\vartheta_{1}$ is given by
\begin{equation}
  \vartheta_{1}
  =\min\Big\{\,\Big|\f{M-\cJ_{\bK}^{n}}{M_{S}-\cJ_{\bK}^{n}}\Big|,\Big|\f{m-\cJ_{\bK}^{n}}{m_{S}-\cJ_{\bK}^{n}}\Big|,1\,\Big\},
\end{equation}
with $m=0$ and $M=1$, and
\begin{equation}
  M_{S}=\max_{\vect{z}\in S}\cJ_{h}^{n}(\vect{z})
  \quad\text{and}\quad
  m_{S}=\min_{\vect{z}\in S}\cJ_{h}^{n}(\vect{z}).  
\end{equation}

In the next step, we ensure realizability of the moments by following the framework of \cite{zhangShu_2010b}, developed to ensure positivity of the pressure when solving the Euler equations of gas dynamics.  
We let $\widetilde{\vect{\cM}}_{h}^{n}=\big(\tilde{\cJ}_{h}^{n},\vect{\cH}_{h}^{n}\big)^{T}$.  
Then, if $\widetilde{\bcM}_{h}^{n}$ lies outside $\cR$ for any quadrature point $\vect{z}_{q}\in S$, i.e., $\gamma(\widetilde{\bcM}_{h}^{n})<0$, there exists an intersection point of the straight line, $\vect{s}_{q}(\xi)$, connecting $\vect{\cM}_{\bK}^{n}\in\cR$ and $\widetilde{\vect{\cM}}_{h}^{n}$ evaluated in the troubled quadrature point $\vect{z}_{q}$, denoted $\widetilde{\vect{\cM}}_{q}^{n}$, and the boundary of $\cR$.  
This line is given by the convex combination 
\begin{equation}
  \vect{s}_{q}(\xi)=\xi\,\widetilde{\vect{\cM}}_{q}^{n}+(1-\xi)\,\bcM_{\bK}^{n},
\end{equation}
where $\xi\in[0,1]$, and the intersection point $\xi_{q}$ is obtained by solving $\gamma(\bs_{q}(\xi))=0$ for $\xi$, using the bisection algorithm.  
We then replace the polynomial representation $\widetilde{\vect{\cM}}_{h}^{n}\to\widehat{\vect{\cM}}_{h}^{n}$, where
\begin{equation}
  \widehat{\vect{\cM}}_{h}^{n}(\vect{z})=\vartheta_{2}\,\widetilde{\vect{\cM}}_{h}^{n}(\vect{z})+(1-\vartheta_{2})\,\vect{\cM}_{\bK}^{n},
  \label{eq:limitMoments}
\end{equation}
and $\vartheta_{2}=\min_{q}\xi_{q}$ is the smallest $\xi$ obtained in the element by considering all the troubled quadrature points.  

We note that the realizability-preserving limiter is conservative; i.e., preserves the cell-average.  