\section{Mathematical Model}
\label{sec:model}

In this section we give a summary of the mathematical model.  

\subsection{Boltzmann Equation}

We consider approximate solutions to the Boltzmann equation for the transport of massless particles through a static material in Cartesian geometry, which, after scaling to dimensionless units, can be written as
\begin{equation}
  \pd{f}{t}+\vect{\ell}\cdot\nabla f
  =\f{1}{\tau}\,\cC(f),
  \label{eq:boltzmann}
\end{equation}
where the distribution function $f\colon(\omega,\varepsilon,\vect{x},t)\in\bbS^{2}\times\bbR^{+}\times\bbR^{3}\times\bbR^{+}\to\bbR^{+}$ gives the number of particles propagating in the direction $\omega\in\bbS^{2}:=\{\,\omega=(\thetaNu,\phiNu)~|~\thetaNu\in[0,\pi],\phiNu\in[0,2\pi)\,\}$, with energy $\varepsilon\in\bbR^{+}$, at position $\vect{x}\in\bbR^{3}$ and time $t\in\bbR^{+}$.  
Here we use spherical momentum space coordinates $(\varepsilon,\omega)$, and the unit vector $\vect{\ell}(\omega)\in\bbR^{3}$ (independent of $\varepsilon$ and $\vect{x}$) is parallel to the particle three-momentum $\vect{p}=\varepsilon\,\vect{\ell}$.  
We also define the energy-position coordinates $\vect{z}:=\{\varepsilon,\vect{x}\}\in\bbR^{+}\times\bbR^{3}$.  
On the right-hand side of Eq.~\eqref{eq:boltzmann}, $\tau$ is the ratio of the particle mean-free path (due to interactions with a background) to some characteristic length scale of the problem.  
In opaque regions, $\tau\ll1$, while for free streaming particles, $\tau\gg1$.  
The collision operator, which models emission, absorption, and isotropic and elastic scattering, is given by
\begin{equation}
  \cC(f)=\xi\,\big(\,f_{0}-f\,\big)
  +(1-\xi)\,\big(\,\f{1}{4\pi}\int_{\bbS^{2}}f\,d\omega-f\,\big),
  \label{eq:collisionTerm}
\end{equation}
where $\xi=\sigma_{\Ab}/\sigma_{\Tot}\in[0,1]$ is the ratio of the absorption opacity $\sigma_{\Ab}\,(\ge0)$ to the total opacity $\sigma_{\Tot}=\sigma_{\Ab}+\sigma_{\Scatt}$, and $\sigma_{\Scatt}\,(\ge0)$ is the scattering opacity.  
In particular, $\xi=1$ models pure emission and absorption, while $\xi=0$ models pure scattering.  
In general, $\sigma_{\Ab}$ and $\sigma_{\Scatt}$ (and $\tau$ and $\xi$) depend on $\vect{z}$.  
The equilibrium distribution function is denoted by $f_{0}(\vect{z})$.  
Here, we consider transport of Fermions (e.g., neutrinos), so the equilibrium distribution function takes the form
\begin{equation}
  f_{0}(\vect{z})=\f{1}{e^{(\varepsilon-\mu(\vect{x}))/T(\vect{x})}+1},  
  \label{eq:fermiDirac}
\end{equation}
where the temperature $T$ and the chemical potential $\mu$ depend on properties of the background.  

\subsection{Angular Moment Equations: Two-Moment Model}

The Boltzmann equation is often too expensive to solve directly.  
Instead, approximate equations for angular moments of the distribution function are solved.  
To this end, we define the angular moments of the distribution function
\begin{equation}
  \big\{\,\cJ,\vect{\cH},\vect{\cK}\,\big\}(\vect{z},t)
  =\f{1}{4\pi}\int_{\bbS^{2}}f(\omega,\vect{z},t)\,\{\,1,\vect{\ell},\vect{\ell}\otimes\vect{\ell}\,\}\,d\omega.  
  \label{eq:angularMoments}
\end{equation}
We refer to $\cJ$ (zeroth moment) as the particle density, $\vect{\cH}$ (first moment) as the particle flux, and $\vect{\cK}$ (second moment) as the stress tensor.  
Note that the moments defined in Eq.~\eqref{eq:angularMoments} are \emph{spectral moments} (depending on energy as well as position and time).  
The \emph{grey moments} (depending only on position and time) are obtained by integration over energy:
\begin{equation}
  \big\{\,J,\vect{H},\vect{K}\,\big\}(\vect{x},t)
  =\int_{\bbR^{+}}\big\{\,\cJ,\vect{\cH},\vect{\cK}\,\big\}(\varepsilon,\vect{x},t)\,\varepsilon^{2}d\varepsilon.  
\end{equation}

Taking the zeroth and first moments of Eq.~\eqref{eq:boltzmann} gives the two-moment model, comprising a system of conservation laws with sources
\begin{equation}
  \pd{\vect{\cM}}{t}+\nabla\cdot\vect{\cF}=\f{1}{\tau}\,\vect{\cC}(\vect{\cM}),
  \label{eq:momentEquations}
\end{equation}
where $\vect{\cM}=(\cJ,\vect{\cH})^{T}$ and $\vect{\cF}=(\vect{\cH},\vect{\cK})^{T}$.  
Components of the fluxes in each coordinate direction are $\vect{\cF}^{i}=\vect{e}_{i}\cdot\vect{\cF}=(\vect{e}_{i}\cdot\vect{\cH},\vect{e}_{i}\cdot\vect{\cK})^{T}$, where $\vect{e}_{i}$ is the unit vector parallel to the $i$th coordinate direction.  
On the right-hand side of Eq.~\eqref{eq:momentEquations}, the source term is
\begin{equation}
  \vect{\cC}(\vect{\cM})=\vect{\eta}-\vect{\cD}\,\vect{\cM}, 
  \label{eq:collisionTermMoments}
\end{equation}
where $\vect{\eta}=(\xi\,f_{0},\vect{0})^{T}$ and $\vect{\cD}=\mbox{diag}(\xi,\vect{I})$, with $\vect{I}$ the identity matrix.  

In order to close the system given by Eq.~\eqref{eq:momentEquations}, the components of the stress tensor $\vect{\cK}$ must be related to the lower moments through a closure procedure.  
To this end, Levermore \cite{levermore_1984} defined the Eddington tensor $\vect{k}=\vect{\cK}/\cJ$ and assumed that the radiation field is symmetric about a preferred direction $\widehat{\vect{h}}=\vect{\cH}/|\vect{\cH}|$ so that
\begin{equation}
  \vect{k}=\f{1}{2}\big[\,\big(1-\chi\big)\,\vect{I}+\big(3\,\chi-1\big)\,\widehat{\vect{h}}\otimes\widehat{\vect{h}}\,\big],
  \label{eq:eddingtonTensor}
\end{equation}
where $\chi=\chi(\cJ,|\vect{\cH}|)$ is the Eddington factor.  
The two-moment model is then closed once the Eddington factor is determined from $\cJ$ and $\vect{\cH}$.  
We will return to the issue of determining the Eddington factor in Section~\ref{sec:algebraicClosure}.  