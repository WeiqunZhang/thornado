\section{Introduction}

Core-collapse supernovae (CCSNe) are the explosions of massive stars that end their lives.
CCSNe are a dominant source of heavy elements and play an important role in many astrophysical phenomena, such as neutron star and black hole formation.  
Furthermore, these explosions occur at energies and densities relevant to address fundamental questions in nuclear, particle, and gravitational physics. 
A solid theoretical framework for the explosion mechanism will illuminate approaches to many import questions in fundamental physics.

One essential part of the explosion mechanism is neutrino transport, which drives the explosion.  
Ideally neutrino transport would be modeled by the Boltzmann transport equation, which is a partial differential equation evolving a phase-space distribution function $f$.
Simulating the neutrino transport is equivalent to finding a solution of the Boltzmann equation for a specific domain and period with acceptable accuracy.

However, solving the multi-dimensional Boltzmann transport equation is expensive.
To balance physical fidelity and computational expediency, an approximate method called the two-moment method has been adopted.
In the two-moment method, the evolved variables are the zeroth and first angular moments of the distribution function $f$: the spectral particle density $\cJ$ and flux $\bcH$, respectively.
However, the Boltzmann transport equation is a nonlinear equation.
Its integral form, which gives the two-moment method system of equations, requires knowledge of the second angular moment $\bcK$ to close this system.
Therefore, an algebraic closure that gives an approximate value of $\bcK$ given $\cJ$ and $\bcH$ is needed.
The two-moment method has been widely applied with different algebraic closures, such as the Minerbo\cite{minerbo_1978} closure(e.g.~{O'Connor} and {Couch}\cite{oConnorCouch_2018}, Pan and et al\cite{pan_etal_2018}, Glas et al\cite{glas_etal_2018}, and Just et al\cite{just_etal_2018} ) and the Levermore\cite{levermore_1984} closure(e.g.~Vartanyan et al\cite{vartanyan_etal_2018}, Cabezon et al\cite{cabezon_etal_2018}, and Kuroda et al\cite{kuroda_etal_2016}). 

Applying the two-moment method alone does not guarantee an affordable solution.
To be precise, neutrinos interact with the background, and the characteristic time scale of these interactions are very small compared to the domination of the CCSN explosion. 
This makes the time step needed by an explicit time integrator prohibitively small. 
On the other hand, solving the moment equations fully implicitly requires inverting globally many band-structured matrices whose sizes depend on the spatial discretization.
Such a global inversion is both expensive and parallelization unfriendly.
To circumvent these challenges, implicit-explicit (IMEX) methods are taken into consideration.
By treating the transport terms in the two-moment equations explicitly and the collision terms implicitly, IMEX methods are subject only to a time step governed by the explicit transport terms, and the matrices to be inverted are block diagonal dominated.
Therefore, IMEX methods require fewer time steps, and the calculation for each step is parallelizable.  

%\ee{Some background on current ``state-of-the-art'' simulations and the approach they take.} 
To employ IMEX schemes for two-moment neutrino transport, two things need to be chosen carefully: a Fermi-Dirac-statistics-preserving algebraic closure for closing the two-moment equations and a convex-invariant, diffusion-accurate IMEX scheme to ensure a physical result.
Because the angular moments are confined.
Since the neutrino distribution function is constrained ($f\in[0,1]$) by the Pauli exclusion principle, its moments as integrals of a constrained function over constrained domain ($\omega\in\bbS^{2}$) are also constrained.
We call the moments satisfying Fermi-Dirac statistics \textit{realizable moments}.
Hence, the algebraic closure should give a realizable $\bcK$, and the well-posedness of closure requires realizable $\cJ$ and $\bcH$.
That's why a Fermi-Dirac statistics satisfied algebraic closure is needed.
Then, the realizability of $\cJ$ and $\bcH$ on each time step requires a convex-invariant IMEX scheme.
Since the realizable moments form a convex set, it is possible to construct a constraint-preserving IMEX scheme for two-moment neutrino transport.
What's more, the physics of neutrino transport in CCSN requires the IMEX scheme to be diffusion-accurate.

\cite{gottlieb_etal_2001} showed standard IMEX scheme can't have an order higher than first without a restrict time step requirement.
To have second-order (or higher-order) accuracy, some correction steps are needed.
Unfortunately, the correction steps can deteriorate the accuracy of the IMEX scheme in the diffusion limit and restrict the time step.
To keep things simple, we focus on IMEX schemes without correction steps and require them to be high-order (second or higher order) in the streaming limit and diffusion-accurate.
We call these IMEX schemes \textit{PD-ARS}.
The study of moment realizability and constraint-preserving, diffusion-accurate IMEX schemes motivate this work.

This paper is organized as follows: Section~\ref{se:Two-MomentModel} discusses the mathematical model, algebraic closures, and the constraints on the moments and algebraic closures imposed by Fermi-Dirac statistics;
Section~\ref{se:SpacialDiscretization} gives a first-order finite-volume spatial discretization and shows how the spatial discretization preserves constraints in an IMEX step;
Section~\ref{se:TimeIntegration} discusses how to use convex combination to construct PD-ARS scheme and two PD-ARS schemes, one with second-order accuracy in the streaming limit and the other with third-order accuracy in the same limit;
Section~\ref{se:NumericalTests} presents the results of the numerical tests, which demonstrate the properties of the PD-ARS schemes; Section~\ref{se:Conclusion} summarizes the achievements of this paper.