\section{Conclusion}\label{se:Conclusion}

We have developed IMEX schemes for the two-moment neutrino transport in \texttt{thornado} code respecting Fermi-Dirac statistics.
The schemes employ algebraic closure based on Fermi-Dirac statistics, a first-order discontinuous Galerkin method, the simple Lax-Friedrichs flux with the convex-invariant time integrator to maintain point-wise realizability of the moments.
Since the realizability-preserving property is obtained from convexity of the realizable domain, it's possible to construct a scheme with high order DG method.
See \cite{chu_etal_2018} for some examples with third-order DG method employed.

In the applications motivating this work, the neutrino distribution function can vary from 0 to 1.
Hence, we have considered the algebraic closure respecting Fermi-Dirac statistics for both low and high occupied condition: different values given by different algebraic closures are compared with the boundary values given by the Fermi-Dirac statistics at different occupation.
Among the seven algebraic closures we considered, Kershaw\cite{kershaw_1976}, Wilson\cite{wilson_1975,leblancWilson_1970}, Levermore\cite{levermore_1984}, Minerbo \cite{minerbo_1978}, Janka 1\cite{janka_1991} Janka 2\cite{janka_1992} and Cernohorsky \& Bludman \cite{cernohorskyBludman_1994}, only Cernohorsky \& Bludman closure respecting Fermi-Dirac statistics for all the occupancy.
So we employ Cernohorsky \& Bludman closure in all the numerical tests in Section~\ref{se:NumericalTests}.

Two PD-ARSs are proposed for the desired IMEX scheme.
The one with SSPRK2 is second-order and strong stability-preserving in the streaming limit while other with SSPRK3 is third-order.
Their properties, both accuracy and convex-invariant, are demonstrated with numerical tests.

In this work, we adopted Cartesian coordinates, linear collision term, non-relativistic, and fixed material background.
More realistic problems of scientific interest, such as with the scattering with energy changes and relativistic effects, are left for future research.