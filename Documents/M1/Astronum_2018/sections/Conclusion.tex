\section{Summary and Outlook}\label{se:Summary}

We have developed IMEX schemes suitable for a two-moment model of neutrino transport that obey Fermi-Dirac statistics.
The scheme employs algebraic closure based on Fermi-Dirac statistics, high-order discontinuous Galerkin methods for spatial discretization, and convex-invariant time integration to maintain realizability of the moments.  
Since the realizable domain is convex and its convexity can be inherited by a convex combination, a scheme having convex combinations as its stages can preserve the realizable domain.
This encouraged us to construct realizability-preserving time integrators, realizability-preserving IMEX schemes, and a method with a realizability-preserving IMEX time integrator and high-order DG method.  

In the applications that motivate this work, the neutrino distribution function can vary from 0 to 1.  
Hence, we have considered algebraic closures based on Fermi-Dirac statistics for both low and high occupancy.  
Among the seven algebraic closures we considered -- Kershaw~\cite{kershaw_1976}, Wilson~\cite{wilson_1975,leblancWilson_1970}, Levermore~\cite{levermore_1984}, Minerbo~\cite{minerbo_1978}, Janka 1~\cite{janka_1991}, Janka 2~\cite{janka_1992}, and Cernohorsky \& Bludman~\cite{cernohorskyBludman_1994} -- only the Cernohorsky \& Bludman closure obeys Fermi-Dirac statistics for all occupancies.  
As a result, we employed the Cernohorsky \& Bludman closure for the neutrino stationary state test in Section~\ref{se: Neutrino Stationary State Test}.
We also ran our code with Minerbo closure, IMEX PC2, IMEX SSP2332 and IMEXRKCB2 schemes.
And the results show that only PD-ARS schemes have stability.
In addition, closures have impact on the simulation result.
Even though RKCB2 with Minerbo closure luckily survived in our test, the result it gives is compromised.
In what way and what degree the results are in fact compromised either by the closure or by a particular correction step for unrealizable moments are difficult to determine and left for further study.


Two PD-ARS schemes are proposed.
The one with SSPRK2 has second-order accuracy while the other with SSPRK3 has third-order accuracy, and both have the strong-stability preserving property in the streaming limit.  
Their accuracy was demonstrated on problems with known smooth solutions in streaming, absorption, and scattering-dominated regimes.
The neutrino transport test with emission, absorption, and isoenergetic scattering through a stationary background, was designed to test the convex-invariance of our PD-ARS schemes. 
The neutrino stationary state test shows that a method combining an algebraic closure based on Fermi-Dirac statistics and convex-invariant time integration is promising for robust CCSN simulation.

In this work, we adopted Cartesian coordinates, a linear collision term, and a fixed material background.
More realistic problems of scientific interest, such as with energy-exchanging scattering and relativistic effects, are left for future research.