\section{Spatial Discretization}\label{se:SpacialDiscretization}

In this section we discretize the two-moment model with a simple first-order finite volume method to illustrate how the closure affects the realizability-preserving property of the method.  
By assuming that the moments at time level $t^{n}$ ($\bcM^{n}$) satisfy the bounds in Eq.~\eqref{eq:MomentsBounds}, our goal is to identify sufficient conditions such that the moments at time level $t^{n+1}$ ($\bcM^{n+1}$) also satisfy the bounds.  
To simply the illustration, we limit the discussion to one spatial dimension and employ a uniform Cartesian mesh.  
(The extension to multiple spatial dimensions and high-order discretization using the discontinuous Galerkin method is given in \cite{chu_etal_2018})

We divide the spatial domain $D$ into $N$ uniform cells and denote the $i$-th cell by $\bK_{i}$, with $i = 1,\ldots,N$; i.e.,
\begin{equation*}
  D = \cup_{i = 1}^{N} \bK_{i} \quad \text{with} \quad
  \bK_{i}=\{\,x : x\in(x_{i-1/2}, x_{i+1/2})\}.
\end{equation*}
Denote the cell width by $\dx = D/N$.  
The cell-average of the moments is defined as
\begin{equation}
  \bcM_{i} = \dfrac{1}{\dx} \int_{\bK_i}\bcM dx.
\end{equation}
Integrating Eq.~\eqref{eq:momentEquations} over $\bK_{i}$ gives
\begin{equation}
  \dfrac{d \bcM_{i}}{d t} = - \dfrac{1}{\dx} \left( \widehat{\bcF}(\bcM_{i},\bcM_{i+1}) -  \widehat{\bcF}(\bcM_{i-1},\bcM_{i})\right) + \f{1}{\tau}\,\cC(\bcM_{i}),
  \label{eq:SemiDiscretizatedMomentEquation}
\end{equation}
where $\widehat{\bcF}(\vect{\cM}_{a},\vect{\cM}_{b})$ is the numerical flux and $\f{1}{\tau}\,\cC(\bcM_{i})$ is the collision term evaluated with $\bcM_{i}$.
In this paper we use the global Lax-Friedrichs flux (setting the largest absolute eigenvalue of the flux Jacobian to one)
\begin{equation}
  \widehat{\bcF}_{\LF}(\vect{\cM}_{a},\vect{\cM}_{b})
  =\f{1}{2}\,\big(\,\bcF(\vect{\cM}_{a})+\bcF(\vect{\cM}_{b})-(\,\vect{\cM}_{b}-\vect{\cM}_{a}\,)\,\big).
  \label{eq:Lax-Friedrichs flux}
\end{equation}
By treating the transport term explicitly with forward Euler and the collision term implicitly with backward Euler, we have
\begin{align}
  \bcM_{i}^{n+1} = \widetilde{\bcM}^{n}_{i} + \f{\dt}{\tau}\,\cC(\bcM^{n+1}_{i}),
  \label{eq:MomentIMEX}
\end{align}
where we have defined
\begin{align}
  \widetilde{\bcM}^{n}_{i} 
  & = \bcM_{i}^{n} - \frac{\dt}{\dx} \left( \widehat{\bcF}_{\LF}(\bcM^{n}_{i},\bcM^{n}_{i+1}) -  \widehat{\bcF}_{\LF}(\bcM^{n}_{i-1},\bcM^{n}_{i})\right)\nonumber \\
  & = (1-\beta)\bcM_{i}^{n} + \beta\left[ \f{1}{2}\left( \bcM^{n}_{i+1}-\bcF(\bcM^{n}_{i+1})\right)  + \f{1}{2}\left( \bcM^{n}_{i-1}+\bcF(\bcM^{n}_{i-1})\right)\right],
  \quad\beta = \frac{\dt}{\dx}.  
\label{eq:widetildeM}
\end{align}

Consider Eq.~\eqref{eq:MomentIMEX} and assume that $\bcM_{i}^{n}$ is realizable for all $i$.  
Lemma~3 in \cite{chu_etal_2018} states that $\bcM^{n+1}_{i}$ is realizable provided $\f{\dt}{\tau} > 0$ and $\widetilde{\bcM}^{n}_{i}$ is realizable.  
In Eq.~\eqref{eq:widetildeM}, if $\beta \in [0,1]$, $\widetilde{\bcM}^{n}_{i}$ is expressed as a convex combination of $\bcM_{i}^{n}$ and the expression in the square brackets on the right-hand side of Eq.~\eqref{eq:widetildeM}.  
It follows that $\widetilde{\bcM}^{n}_{i}$ is realizable if the expression inside the square brackets is realizable.  
By Lemma~2 in \cite{chu_etal_2018}, the expression in square brackets is realizable for a distribution satisfying $f\in[0,1]$.  
For the two-moment model considered here, realizability depends in the algebraic closure.  
Specifically, if $\bcM_{i}^{n}$ is realizable and the Eddington factor satisfies the bounds in Eq.~\eqref{eq:eddingtonFactorBounds}, then $\bcM^{n+1}_{i}$ is realizable provided $\beta \in [0,1]$.  
Thus, realizability of $\bcM^{n+1}_{i}$ requires both a closure based on Fermi-Dirac statistics and a CFL condition $\dt\le\dx$.  