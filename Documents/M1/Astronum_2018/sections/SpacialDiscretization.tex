\section{Spacial Discretization}\label{se:SpacialDiscretization}

Divide the spatial domain $D$ into $N$ cells and denote the cell as $I_{i}$ with $i = 1,\ldots,N$, so that
\begin{equation*}
D = \cup_{i = 1}^{N} I_{i}.
\end{equation*}
Each of the cell has a volume/length of $\dx$.
Then we have cell average moment:
\begin{equation}
\bcM_{i} = \dfrac{1}{\dx} \int_{I_i}\bcM dx.
\end{equation}
Integrate Eq.~\eqref{eq:momentEquations} for each cell $I_{i}$ to have
\begin{equation}
\dfrac{d \bcM_{i}}{d t} = - \dfrac{1}{\dx} \left( \widehat{\bcF}(\bcM_{i},\bcM_{i+1}) -  \widehat{\bcF}(\bcM_{i-1},\bcM_{i})\right) + \f{1}{\tau}\,\cC(\bcM_{i}),
\end{equation}
where $\widehat{\bcF}(\vect{\cM}_{a},\vect{\cM}_{b})$ is numerical flux and $\f{1}{\tau}\,\cC(\bcM_{i})$ is the integrated collision term.
In this paper we use the simple Lax-Friedrichs flux that given by
\begin{equation}
  \widehat{\bcF}_{LF}(\vect{\cM}_{a},\vect{\cM}_{b})
  =\f{1}{2}\,\big(\,\bcF(\vect{\cM}_{a})+\bcF(\vect{\cM}_{b})-(\,\vect{\cM}_{b}-\vect{\cM}_{a}\,)\,\big).
\end{equation}
Therefore, by treating the flux term explicitly and the collision term implicitly, we have
\begin{align*}
\bcM_{i}^{n+1} = \widetilde{\bcM}^{n}_{i} + \f{\dt}{\tau}\,\cC(\bcM^{n+1}_{i})
\end{align*}
with
\begin{align*}
\widetilde{\bcM}^{n}_{i} 
& = \bcM_{i}^{n} - \dfrac{\dt}{\dx} \left( \widehat{\bcF}(\bcM^{n}_{i},\bcM^{n}_{i+1}) -  \widehat{\bcF}(\bcM^{n}_{i-1},\bcM^{n}_{i})\right)\nonumber \\
& = (1-\beta)\bcM_{i}^{n} + \beta\left[ \f{1}{2}\left( \bcM^{n}_{i+1}-\bcF(\bcM^{n}_{i+1})\right)  + \f{1}{2}\left( \bcM^{n}_{i-1}+\bcF(\bcM^{n}_{i-1})\right)\right]
\end{align*}
where $\beta := \dfrac{\dt}{\dx}$.
Assume $\bcM_{i}^{n}$ is realizable for all $i$. 
Lemma 3 in \cite{chu_2018} says $\bcM^{n+1}_{i}$ remains realizable given $\f{\dt}{\tau} > 0$ and $\widetilde{\bcM}^{n}_{i}$ is realizable.
Substitute the Lax-Friedrichs flux formula in and $\widetilde{\bcM}^{n}_{i}$ becomes a convex combination of $\bcM_{i}^{n}$ and the square bracket term if $\beta \in [0,1]$.
So it is realizable if $\beta \in [0,1]$ and the square bracket term is realizable.
Given by Lemma 2 in \cite{chu_2018}, the square bracket term is realizable.
Therefore, $\bcM^{n+1}_{i}$ remains realizable given $\beta \in [0,1]$.
Note that $\beta \in [0,1]$ is also the CFL condition of the Forward-Euler method employing on a transport only Boltzmann equation.
