\clearpage
\section{Time Integration} \label{se:TimeIntegration}

Suppose that an algebraic closure based on Fermi-Dirac statistics is used (i.e., the Eddington factor satisfies Eq.~\eqref{eq:eddingtonFactorBounds}).
Here we consider the construction of an Implicit-Explicit (IMEX) time integration scheme which maintains the bounds in Eq.~\eqref{eq:MomentsBounds}.  
The semi-discretization of the two-moment model results in a system of ordinary differential equations of the form
\begin{equation}
  \dfrac{d \vect{u}}{d t} = \vect{\cT}(\vect{u}) + \f{ 1}{\tau}\,\vect{\cQ}(\vect{u}),
\end{equation}
where the solution vector
\begin{equation}
  \vect{u}(t) = \left( \bcM_{1}(t),\ldots,\bcM_{N}(t)\right) ^{T}
\end{equation}
is the collection of all cell-averaged moments, $\vect{\cT}$ is the transport operator, corresponding to the first term on the right-hand side of Eq.~\eqref{eq:SemiDiscretizatedMomentEquation}, and $\f{1}{\tau}\,\vect{\cQ}$ is the collision operator, corresponding to the second term on the right-hand side of Eq.~\eqref{eq:SemiDiscretizatedMomentEquation}.  

Since the set of realizable moments is convex, convex-invariant schemes, which maintain states in a convex set, can be used to design realizability-preserving methods for the two-moment model. 
Ideally, the scheme should also be high-order accurate and work well in the asymptotic diffusion limit (characterized by frequent collisions and long time scales).  
The following discussion considers the construction of such convex-invariant schemes.  

\subsection{Standard IMEX Schemes}

Treating the transport operator explicitly and the collision operator implicitly, a standard $s$-stage IMEX scheme takes the following form \cite{pareschiRusso_2005}: 
\begin{align}
  \vect{u}^{(i)}
  &=\vect{u}^{n}
  +\dt\sum_{j=1}^{i-1}\tilde{a}_{ij}\,\vect{\cT}(\vect{u}^{(j)})
  +\dt\sum_{j=1}^{i}a_{ij}\,\f{1}{\tau}\,\vect{\cQ}(\vect{u}^{(j)}),
  \quad i=1,\ldots,s, \label{imexStages} \\
  \vect{u}^{n+1}
  &=\vect{u}^{n}
  +\dt\sum_{i=1}^{s}\tilde{w}_{i}\,\vect{\cT}(\vect{u}^{(i)})
  +\dt\sum_{i=1}^{s}w_{i}\,\f{1}{\tau}\,\vect{\cQ}(\vect{u}^{(i)}), \label{imexIntermediate} 
\end{align}
where $(\tilde{a}_{ij})$ and $(a_{ij})$, coefficients of the $i$-th stage, are the elements of matrices $\tilde{A}$ and $A$, respectively.
Matrices $\tilde{A}$ and $A$ are both lower triangular matrices whose other elements are all zero.
The vectors $\tilde{\vect{w}}=(\tilde{w}_{1},\ldots,\tilde{w}_{s})^{T}$ and $\vect{w}=(w_{1},\ldots,w_{s})^{T}$ are the weights in the assembly step in Eq.~\eqref{imexIntermediate}.
These coefficients and weights must satisfy certain order conditions for consistency, accuracy and other properties.
  
For second-order temporal accuracy, the following conditions are required \cite{hairer_1981}:
\begin{equation}
  \sum_{i=1}^{s}\tilde{w}_{i}=\sum_{i=1}^{s}w_{i}=1,
  \label{orderConditions1}
\end{equation}
and
\begin{equation}
  \sum_{i=1}^{s}\tilde{w}_{i}\,\tilde{c}_{i}
  =\sum_{i=1}^{s}\tilde{w}_{i}\,c_{i}
  =\sum_{i=1}^{s}w_{i}\,\tilde{c}_{i}
  =\sum_{i=1}^{s}w_{i}\,c_{i}=\f{1}{2}, 
  \label{orderConditions2}
\end{equation}
where $\tilde{c}_{i} = \sum_{j=1}^{s}\tilde{a}_{ij}$ and $c_{i}=\sum_{j=1}^{s}a_{ij}$.

For globally stiffly accuracy (GSA), the coefficients have the following relationship \cite{dimarcoPareschi2013}:
\begin{equation}
a_{si}=w_{i}, \quad \tilde{a}_{si}=\tilde{w}_{i}, \quad \text{for} \quad i=1,\ldots,s.
\end{equation}
It makes the assembly step the same as the $s$th-stage, $\vect{u}^{n+1} = \vect{u}^{(s)}$.
Therefore, GSA condition, if it is achievable, is also a simplifying condition beyond stiffly accurate condition.

The IMEX schemes are usually classified by the structure of matrix $A$, the implicit tableau:
if $A$ is invertible, the IMEX scheme is type~A \cite{pareschiRusso_2005};
if $a_{i1} = 0$ for $i=1,\ldots,s$ and $w_{1} = 0$, the IMEX scheme is type~ARS \cite{ascher_etal_1997,pareschiRusso_2005}. 

\subsection{Convex-Invariant Implicit-Explicit Schemes}

Beyond certain accuracy and GSA requirements, a convex-invariant IMEX scheme must satisfy additional constraints.
Our idea is finding the constraints on $a_{ij}$, $\tilde{a}_{ij}$, $\tilde{w}_{i}$, and $w_{i}$ that enable each $\vect{u}^{(i)}$ and $\vect{u}^{n+1}$ be convex combination of realizable moments:

Following Hu et al\cite{hu_etal_2018}, the stage values in Eq.~\eqref{imexStages} can be rewritten as
\begin{equation}
  \vect{u}^{(i)}
  =\sum_{j=0}^{i-1}c_{ij}\Big[\,\vect{u}^{(j)}+\hat{c}_{ij}\,\dt\,\vect{\cT}(\vect{u}^{(j)})\,\Big]
  +a_{ii}\,\dt\,\f{1}{\tau}\,\vect{\cQ}(\vect{u}^{(i)}),\quad i=1,\ldots,s,
  \label{eq:imexStagesRewrite}
\end{equation}
with $c_{ij}$ and $\hat{c}_{ij}$ be some expression of $a_{ij}$ and $\tilde{a}_{ij}$.
For example, type~ARS IMEX schemes have $c_{ij}$ and $\hat{c}_{ij}$ be given by
    \begin{equation}
     \begin{aligned}
      c_{i0} &= 1-\sum_{j=2}^{i-1}\sum_{l=j}^{i-1}a_{il}b_{lj}, \quad &
      c_{ij} &= \sum_{l=j}^{i-1}a_{il}b_{lj}, \\
      \tilde{c}_{i0} &= \tilde{a}_{i1}+\sum_{j=2}^{i-1}a_{ij}\tilde{b}_{j1}, \quad &
      \tilde{c}_{ij} &= \tilde{a}_{ij}+\sum_{l=j+1}^{i-1}a_{il}\tilde{b}_{lj},  
     \end{aligned}
     \label{eq:positivityCoefficientsARS}
    \end{equation}
    \begin{equation}
      b_{ii} = \f{1}{a_{ii}}, \quad
      b_{ij} = -\f{1}{a_{ii}}\sum_{l=j}^{i-1}a_{il}b_{lj}, \quad
      \tilde{b}_{ij} = -\f{1}{a_{ii}}\Big(\tilde{a}_{ij}+\sum_{l=j+1}^{i-1}a_{il}\tilde{b}_{lj}\Big).  
    \end{equation}
Note that $c_{i1}=\tilde{c}_{i1}=0$ in Eq.~\eqref{eq:positivityCoefficientsARS} so that $\sum_{j=0}^{i-1}c_{ij}=1$.

GSA gives $\vect{u}^{n+1} = \vect{u}^{(s)}$.
Then using those Lemma proved in \cite{chu_etal_2018}, we can find those constraints, such as Eq.~\eqref{eq:convexInvariant} for type~ARS, that $c_{ij}$ and $\hat{c}_{ij}$ (or $a_{ij}$ and $\tilde{a}_{ij}$) need to satisfy to ensure $\vect{u}^{(i)}$, Eq.~\eqref{eq:imexStagesRewrite}, be realizable.
Hence $\vect{u}^{(i)}$ and $\vect{u}^{n+1}$ are realizable and the method with this convex-invariant IMEX scheme is realizability-preserving.

When the convex-invariant constraints are satisfied, the first (sum) term on the right-hand side of Eq.\eqref{eq:imexStagesRewrite} is a convex combination of multiple explicit Euler schemes with time step $\hat{c}_{ij}\,\dt$.
Each of the explicit Euler schemes has a time step condition that ensures its realizability: $\hat{c}_{ij}\,\dt\leq$ the Courant-Friedrichs-Lewy (CFL) condition.
As a result, the time step of the IMEX scheme is subject to the most restrict time step condition:
\begin{equation}
\max(\hat{c}_{ij})\,\dt \leq \text{CFL condition}.
\end{equation}

Unfortunately, no coefficients can be found for standard IMEX schemes under the convex-invariant constraints and the accuracy constraints for 2nd-order or higher. 
To circumvent this problem, some correction steps are introduced after the assembly step, Eq.~\eqref{imexIntermediate}, such as Hu et al.\cite{hu_etal_2018}.
However, the correction steps impose extra constraints for realizability on time step.
In the circumstance that we have, correction steps can ruin the efficiency we gain from the IMEX scheme.
To keep things simple, we aim for 1st-order in general, high-order in streaming limit, diffusion accurate, convex-invariant IMEX schemes.

\subsection{Diffusion Accurate, Convex-Invariant Implicit-Explicit Schemes}

Diffusion accurate is another important property the time integration should have.
In the diffusion limit, the distribution function is nearly isotropic so that $\vect{\cK}\approx\f{1}{3}\,\cJ\,\vect{I}$ and $\vect{\cH}\approx-\f{1}{3}\,\tau\,\nabla\cJ$.
The moment behavior are governed by (e.g., \cite{jinLevermore_1996})
\begin{equation}
  \pd{\cJ}{t} + \nabla\cdot\vect{\cH} = 0
  \quad\text{and}\quad
  \vect{\cH} = - \tau\,\nabla\cdot\vect{\cK}.  
  \label{eq:diffusionLimit}
\end{equation}
In the context of IMEX schemes, the above relationships imply that the relation $A\,\vec{\vect{\cH}}=-\f{1}{3}\,\tau\,\tilde{A}\,\nabla\vec{\cJ}$ should hold. 
That's
\begin{equation}
   \vect{e}_{i}^{T}A^{-1}\tilde{A}\,\vect{e} = 1, \quad i=1,\ldots,s,
\end{equation}
where $\vect{e}_{i}$ is the $i$th column of the $s\times s$ identity matrix.
It can be rewritten as 
\begin{equation}
c=\tilde{c}.
\end{equation}
We have proved in \cite{chu_etal_2018} that only type~ARS IMEX schemes can be diffusion accurate and convex-invariant at the same time.
Another short proof is that type~A IMEX schemes have $\tilde{c}_1 = 0$ while $c_i \neq 0$.

\subsection{PD-ARS schemes}

Combining the conditions we have, the 1st-order in general, high-order in streaming limit, diffusion accurate, convex-invariant, GSA IMEX schemes we need are type~ARS IMEX schemes satisfying the following constraints:
\begin{enumerate}
    \item Consistency of the implicit coefficients:
    \begin{equation}
      \sum_{i=1}^{s}w_{i}=1.
    \end{equation}
    \item High-order accuracy in the streaming limit, 2nd-order for example:
    \begin{equation}
      \sum_{i=1}^{s}\tilde{w}_{i}=1
      \quad\text{and}\quad
      \sum_{i=1}^{s}\tilde{w}_{i}\,\tilde{c}_{i}=\f{1}{2}.
      \label{eq:orderConditionsEx}
    \end{equation}
    \item Diffusion accurate:
    \begin{equation}
      c=\tilde{c}.
      \label{eq:diffusionCondition}
    \end{equation}
    \item Convex-invariant:
    \begin{align}
      &a_{ii}>0, \quad c_{i0},\tilde{c}_{i0}\ge0, \quad \text{for} \quad i=2,\ldots,s, \nonumber \\
      &\text{and} \quad c_{ij},\tilde{c}_{ij}\ge0, \quad \text{for} \quad i=3,\ldots,s, \quad\text{and}\quad j=2,\ldots,i-1.  
      \label{eq:convexInvariant}
    \end{align}
    with $\sum_{j=0}^{i-1}c_{ij}=1$, for $i=1,\ldots,s$, and $c_{\Sch}:=\min_{\substack{i = 2,\ldots,s \\ 
                  j = 0,2,\ldots,i-1}}\,\f{1}{\hat{c}_{ij}}>0$.
    \item Less than five stages ($s\le4$) \label{cod:statges}.
    \item Globally stiffly accurate: $a_{si}=w_{i}$ and $\tilde{a}_{si}=\tilde{w}_{i},\quad i=1,\ldots,s$. 
\end{enumerate}  
The \ref{cod:statges} constraint is given by efficiency consideration.
We call the IMEX scheme satisfying the above conditions {PD-ARS}. (Definition 3 in \cite{chu_etal_2018}.)
We give two optimized PD-ARS schemes here: PD-ARS with SSPRK2 and  PD-ARS with SSPRK3. 
They have second- and third-order accuracy in the streaming limit, respectively.
\subsubsection{PD-ARS with SSPRK2}
Here we give the optimized PD-ARS with SSPRK2 in the standard double Butcher tableau form: explicit tableau on the left ($\tilde{A}$), implicit tableau on the right ($A$):
\begin{equation}
  \begin{array}{c | c c c}
  	0 & 0   & 0 & 0 \\
  	1 & 1   & 0 & 0 \\
  	1 & 1/2 & 1/2 & 0 \\ \hline
  	  & 1/2 & 1/2 & 0
  \end{array}
  \qquad
  \begin{array}{c | c c c}
  	0 & 0 & 0            & 0            \\
  	1 & 0 & 1            & 0            \\
  	1 & 0 & 1/2 & 1/2 \\ \hline
  	  & 0 & 1/2 & 1/2
  \end{array}
\end{equation}
For this scheme, $c_{\mbox{\tiny Sch}}= 1$ and only two implicit solver are needed for each time step.
\subsubsection{PD-ARS with SSPRK3}
Here we give the optimized PD-ARS with SSPRK3.
Its standard double Butcher tableau are (explicit tableau on the left, implicit tableau on the right)
\begin{equation}
  \begin{array}{c | c c c c}
  	    &     &     &     &  \\
  	 1  & 1   &     &     &  \\
  	1/2 & 1/4 & 1/4 &  \\
  	 1  & 1/6 & 1/6 & 2/3 &  \\ \hline
  	    & 1/6 & 1/6 & 2/3 &
  \end{array}
  \qquad
  \begin{array}{c | c c c c}
  	0 & 0 & 0            & 0            \\
  	1 & 0 & 1            & 0            \\
  	1/2 & 0 & 1/4 & 1/4 \\ 
  	1 & 0 & 1/6 & 1/6 & 2/3\\\hline
  	  & 0 & 1/6 & 1/6 & 2/3
  \end{array}
\end{equation}
For this schemes, $c_{\mbox{\tiny Sch}}= 1$.
Three implicit solver are taken for each time step.
Since PD-ARS with SSPRK3 is as accurate as PD-ARS with SSPRK2 in collision involved region, see Section~\ref{se:NumericalTests} for the plots, it may not offer any advantage.
