\section{Time Integration} \label{se:TimeIntegration}

Suppose we have an algebraic closure based on Fermi-Dirac statistic (whose Eddington factor satisfies Eq.~\eqref{eq:eddingtonFactorBounds}).
We show how to construct a time integration which satisfies the bounds on moment after each time step in this section.
The semi-discretization of the moment equations, Eq.~\eqref{eq:momentEquations}, results in a system of ordinary differential equations of the form
\begin{equation}
  \dfrac{d \vect{u}}{d t} = \vect{\cT}(\vect{u}) + \f{1}{\tau}\,\vect{\cQ}(\vect{u}),
\end{equation}
where
\begin{equation}
\vect{u}(t) = \left( \bcM_{1}(t),\ldots,\bcM_{N}(t)\right) ^{T}
\end{equation}
is the collection of all cell-average moments, $\vect{\cT}$ is the transport operator corresponding to the first term on the right-hand side of Eq.~\eqref{eq:SemiDiscretizatedMomentEquation}, and $\f{1}{\tau}\,\vect{\cQ}$ corresponds to the collision operator, the second term on the right-hand side of Eq.~\eqref{eq:SemiDiscretizatedMomentEquation}.

Since the realizable $\bcM$ define a convex set in $\cJ$-$\bcH$ space, convex-invariant schemes, which remain in their initial convex set, can be used to design realizability-preserving schemes for the two-moment model. 
The following discussion shows how to construct a convex-invariant scheme.

\subsection{Standard Implicit-Explicit Scheme}
Treating $\vect{\cT}$ explicitly and $\f{1}{\tau}\,\vect{\cQ}$ implicitly, a standard $s$-stage IMEX scheme takes the following form \cite{pareschiRusso_2005}
\begin{align}
  \vect{u}^{(i)}
  &=\vect{u}^{n}
  +\dt\sum_{j=1}^{i-1}\tilde{a}_{ij}\,\vect{\cT}(\vect{u}^{(j)})
  +\dt\sum_{j=1}^{i}a_{ij}\,\f{1}{\tau}\,\vect{\cQ}(\vect{u}^{(j)}),
  \quad i=1,\ldots,s, \label{imexStages} \\
  \vect{u}^{n+1}
  &=\vect{u}^{n}
  +\dt\sum_{i=1}^{s}\tilde{w}_{i}\,\vect{\cT}(\vect{u}^{(i)})
  +\dt\sum_{i=1}^{s}w_{i}\,\f{1}{\tau}\,\vect{\cQ}(\vect{u}^{(i)}), \label{imexIntermediate} 
\end{align}
where  $(\tilde{a}_{ij})$ and $(a_{ij})$ are coefficients of the $i$-th stage and they are the elements of matrices $\tilde{A}$ and $A$, respectively.
The vectors $\tilde{\vect{w}}=(\tilde{w}_{1},\ldots,\tilde{w}_{s})^{T}$ and $\vect{w}=(w_{1},\ldots,w_{s})^{T}$ are the weight vectors.
These coefficients and weights have to satisfy certain condition for an accurate, convergent and stable scheme.
For instance, to have second-order temporal accuracy, the following conditions are required:
\begin{equation}
  \sum_{i=1}^{s}\tilde{w}_{i}=\sum_{i=1}^{s}w_{i}=1,
  \label{orderConditions1}
\end{equation}
and
\begin{equation}
  \sum_{i=1}^{s}\tilde{w}_{i}\,\tilde{c}_{i}
  =\sum_{i=1}^{s}\tilde{w}_{i}\,c_{i}
  =\sum_{i=1}^{s}w_{i}\,\tilde{c}_{i}
  =\sum_{i=1}^{s}w_{i}\,c_{i}=\f{1}{2}, 
  \label{orderConditions2}
\end{equation}
where $\tilde{c}_{i} = \sum_{j}^{s}\tilde{a}_{ij}$ and $c_{i}=\sum_{j}^{s}a_{ij}$.

\subsection{A Constraint-Preserving Implicit-Explicit Scheme}
Beyond the accuracy requirement, a convex-invariant IMEX scheme has extra constraints to satisfy.
The problem left is to find the additional constraints on $(\tilde{a}_{ij})$, $(a_{ij})$, $\tilde{w}_{i}$, and $w_{i}$.
Following Hu et al\cite{hu_etal_2018}, the stage values in Eq.~\eqref{imexStages} can be rewritten as
\begin{equation}
  \vect{u}^{(i)}
  =\sum_{j=0}^{i-1}c_{ij}\Big[\,\vect{u}^{(j)}+\hat{c}_{ij}\,\dt\,\vect{\cT}(\vect{u}^{(j)})\,\Big]
  +a_{ii}\,\dt\,\f{1}{\tau}\,\vect{\cQ}(\vect{u}^{(i)}),\quad i=1,\ldots,s,
  \label{eq:imexStagesRewrite}
\end{equation}
with $c_{ij}$ and $\hat{c}_{ij}$ be given by $a_{ij}$ and $\tilde{a}_{ij}$.

It can be proven that the conditions we need for a simple convex-invariant IMEX scheme are: \cite{chu_etal_2018}
\begin{enumerate}
    \item Consistency of the implicit coefficients:
    \begin{equation}
      \sum_{i=1}^{s}w_{i}=1.
    \end{equation}
    \item Second-order accuracy in the streaming limit:
    \begin{equation}
      \sum_{i=1}^{s}\tilde{w}_{i}=1
      \quad\text{and}\quad
      \sum_{i=1}^{s}\tilde{w}_{i}\,\tilde{c}_{i}=\f{1}{2},
      \label{eq:orderConditionsEx}
    \end{equation}
    \item Convex-invariant:
    \begin{align}
      &a_{i1} = 0, \quad \text{for} \quad i=1,\ldots,s, \nonumber \\
      &a_{ii}>0, \quad c_{i0},\tilde{c}_{i0}\ge0, \quad \text{for} \quad i=2,\ldots,s, \nonumber \\
      &\text{and} \quad c_{ij},\tilde{c}_{ij}\ge0, \quad \text{for} \quad i=3,\ldots,s, \quad\text{and}\quad j=2,\ldots,i-1.  
    \end{align}
    with $\sum_{j=0}^{i-1}c_{ij}=1$, for $i=1,\ldots,s$, and $c_{\Sch}=\min_{\substack{i = 2,\ldots,s \\ 
                  j = 0,2,\ldots,i-1}}\,\f{1}{\hat{c}_{ij}}>0$.
    And $c_{ij}$ and $\hat{c}_{ij}$ be given by
    \begin{equation}
     \begin{aligned}
      c_{i0} &= 1-\sum_{j=2}^{i-1}\sum_{l=j}^{i-1}a_{il}b_{lj}, \quad &
      c_{ij} &= \sum_{l=j}^{i-1}a_{il}b_{lj} \\
      \tilde{c}_{i0} &= \tilde{a}_{i1}+\sum_{j=2}^{i-1}a_{ij}\tilde{b}_{j1}, \quad &
      \tilde{c}_{ij} &= \tilde{a}_{ij}+\sum_{l=j+1}^{i-1}a_{il}\tilde{b}_{lj}.  
     \end{aligned}
     \label{eq:positivityCoefficientsARS}
    \end{equation}
    \begin{equation}
      b_{ii} = \f{1}{a_{ii}}, \quad
      b_{ij} = -\f{1}{a_{ii}}\sum_{l=j}^{i-1}a_{il}b_{lj}, \quad
      \tilde{b}_{ij} = -\f{1}{a_{ii}}\Big(\tilde{a}_{ij}+\sum_{l=j+1}^{i-1}a_{il}\tilde{b}_{lj}\Big).  
    \end{equation}
    Note that $c_{i1}=\tilde{c}_{i1}=0$ in Eq.~\eqref{eq:positivityCoefficientsARS} so that $\sum_{j=0}^{i-1}c_{ij}=1$.
    \item Well-behaved in the diffusion limit:
    \begin{equation}
      \vect{e}_{i}^{T}A^{-1}\tilde{A}\,\vect{e} = 1, \quad i=2,\ldots,s,
      \label{eq:diffusionCondition}
    \end{equation}
    where $\vect{e}_{i}$ is the $i$th column of the $s\times s$ identity matrix.
    \item Less than four stages ($s\le3$).
    \item Globally stiffly accurate: $a_{si}=w_{i}$ and $\tilde{a}_{si}=\tilde{w}_{i},\quad i=1,\ldots,s$. (GSA)
\end{enumerate}  
We call the IMEX scheme satisfying the above conditions {PD-ARS}. (Definition 3 in \cite{chu_etal_2018}.)
We give two optimized PD-ARS schemes here: PD-ARS with SSPRK2 and  PD-ARS with SSPRK3. 
They have second- and third-order accuracy in the streaming limit, respectively.
\subsubsection{PD-ARS with SSPRK2}
Here we give a family of PD-ARS with SSPRK2 in the standard double Butcher tableau form: explicit tableau on the left ($\tilde{A}$), implicit tableau on the right ($A$):
\begin{equation}
  \begin{array}{c | c c c}
  	0 & 0   & 0 & 0 \\
  	1 & 1   & 0 & 0 \\
  	1 & 1/2 & 1/2 & 0 \\ \hline
  	  & 1/2 & 1/2 & 0
  \end{array}
  \qquad
  \begin{array}{c | c c c}
  	0 & 0 & 0            & 0            \\
  	1 & 0 & 1            & 0            \\
  	1 & 0 & 1/2-\epsilon & 1/2+\epsilon \\ \hline
  	  & 0 & 1/2-\epsilon & 1/2+\epsilon
  \end{array}
\end{equation}
and $\epsilon \in [0, 1/2)$.
For these schemes, $c_{\mbox{\tiny Sch}}= 1 - 2\epsilon$.
The optimized scheme ($c_{\mbox{\tiny Sch}}= 1$) is given by $\epsilon = 0$.
We chose $\epsilon = 0.1$ for the numerical test in Section~\ref{se:NumericalTests}.

\subsubsection{PD-ARS with SSPRK3}
Here we give the optimized PD-ARS with SSPRK3.
Its standard double Butcher tableau are (explicit tableau on the left, implicit tableau on the right)
\begin{equation}
  \begin{array}{c | c c c c}
  	    &     &     &     &  \\
  	 1  & 1   &     &     &  \\
  	1/2 & 1/4 & 1/4 &  \\
  	 1  & 1/6 & 1/6 & 2/3 &  \\ \hline
  	    & 1/6 & 1/6 & 2/3 &
  \end{array}
  \qquad
  \begin{array}{c | c c c c}
  	0 & 0 & 0            & 0            \\
  	1 & 0 & 1            & 0            \\
  	1/2 & 0 & 1/4 & 1/4 \\ 
  	1 & 0 & 1/6 & 1/6 & 2/3\\\hline
  	  & 0 & 1/6 & 1/6 & 2/3
  \end{array}
\end{equation}
For this schemes, $c_{\mbox{\tiny Sch}}= 1$.