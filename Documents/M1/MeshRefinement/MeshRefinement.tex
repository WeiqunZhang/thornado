\documentclass[10pt]{article}

\usepackage{amsfonts}
\usepackage{amsmath}
\usepackage{amssymb}
\usepackage{amsthm}
\usepackage{booktabs}
\usepackage{mathrsfs}
\usepackage{graphicx}
\usepackage{cite}
\usepackage{times}
\usepackage{url}
\usepackage{hyperref}
\usepackage{lineno}
\usepackage{yhmath}
\usepackage{natbib}
\usepackage{../../definitions}
\hypersetup{
  bookmarksnumbered = true,
  bookmarksopen=false,
  pdfborder=0 0 0,         % make all links invisible, so the pdf looks good when printed
  pdffitwindow=true,      % window fit to page when opened
  pdfnewwindow=true, % links in new window
  colorlinks=true,           % false: boxed links; true: colored links
  linkcolor=blue,            % color of internal links
  citecolor=magenta,    % color of links to bibliography
  filecolor=magenta,     % color of file links
  urlcolor=cyan              % color of external links
}

\newcommand{\ee}[1]{{\color{blue} EE:~#1}}
\newcommand{\Lo}{\textsc{L}}
\newcommand{\Hi}{\textsc{H}}
\newcommand{\trans}{\textsc{T}}
\newcommand{\dx}{\Delta x}

\newtheorem{define}{Definition}
\newtheorem{lemma}{Lemma}
\newtheorem{prop}{Proposition}
\newtheorem{rem}{Remark}
\newtheorem{theorem}{Theorem}

\begin{document}

\title{Mesh Refinement and Derefinement of Two-Moment Radiation Fields}
\author{Eirik Endeve et al.}

\maketitle

\begin{abstract}
  We consider refinement and coarsening of polynomial representations.
  We basically follow Schaal et al.~\cite{schaal_etal_2015}.
\end{abstract}

\tableofcontents

\section{Preliminaries}

\subsection{Coarse and Fine Elements}

Consider the $d$-dimensional (coarse) element $\bK$
\begin{equation}
  \vect{K} = \big\{\,\vect{x} : x^{i} \in K^{i} := (x_{\Lo}^{i},x_{\Hi}^{i}), i = 1,\ldots,d \,\big\}.
\end{equation}
We also define the unit reference element $\vect{I}$
\begin{equation}
  \vect{I} = \big\{\,\vect{\eta} : \eta^{i} \in (-1/2,+1/2), i = 1,\ldots,d \,\big\},
\end{equation}
and the standard mapping $x^{i}(\eta^{i})=x_{\Lo}^{i}+\eta^{i}\,\dx^{i}$, where $\dx^{i}=x_{\Hi}^{i}-x_{\Lo}^{i}$.
We also define $|\vect{K}|=\prod_{i=1}^{d}\dx^{i}$.

The element $\vect{K}$ is to be refined into $N$ (finer) elemets $\vect{k}_{(j)}$
\begin{equation}
  \vect{k}_{(j)} = \big\{\,\vect{x} : x^{i} \in k_{(j)}^{i} := (x_{\Lo,(j)}^{i},x_{\Hi,(j)}^{i}), i = 1,\ldots,d\,\big\},
\end{equation}
where
\begin{equation}
  \vect{K} = \cup_{j=1}^{N}\vect{k}_{(j)}.
\end{equation}
We will also consider derefinement (coarsening).
The reference element associated with $\vect{k}_{(j)}$ is denoted $\vect{i}_{(j)}$
\begin{equation}
  \vect{i}_{(j)} = \big\{\,\vect{\xi} : \xi^{i} \in (-1/2,+1/2), i = 1,\ldots,d \,\big\},
\end{equation}
with mapping $x^{i}(\xi^{i})=x_{\Lo,(j)}^{i}+\xi^{i}\,\dx_{(j)}^{i}$, where $\dx_{(j)}^{i}=x_{\Hi,(j)}^{i}-x_{\Lo,(j)}^{i}$.
We also let $|\vect{k}_{(j)}|=\prod_{i=1}^{d}\dx_{(j)}^{i}$.

\subsection{Approximation Spaces}

On $\vect{K}$, we denote the approximation space by
\begin{equation}
  \mathbb{V}^{k} = \big\{\,v_{h} : v_{h}\big|_{\vect{K}}\in\mathbb{Q}^{k}(\vect{K})\,\big\}
\end{equation}
where $\mathbb{Q}^{k}$ is constructed from the tensor product of one-dimensional polynomials of degree at most $k$.
Similarly, the approximation space on $\vect{k}_{(j)}$ is
\begin{equation}
  \mathbb{V}_{(j)}^{k} = \big\{\,v_{h} : v_{h}\big|_{\vect{k}_{(j)}}\in\mathbb{Q}^{k}(\vect{k}_{(j)})\,\big\}.
\end{equation}

\subsection{Polynomial Representations}

On the coarse element $\vect{K}$, the physical (conserved) variable $U$ is approximated by $U_{h}$
\begin{equation}
  U_{h}(\vect{x}) = \sum_{i=1}^{M}U_{i}\,\ell_{i}(\vect{x}),
\end{equation}
where $\ell_{i}\in\mathbb{V}^{k}$.
The vector of expansion coefficients in the coarse element $\vect{K}$ is denoted $\vect{U}=(U_{1},\ldots,U_{M})^{\trans}$.
Similarly, in each fine element $\vect{k}_{(j)}$, the same variable is approximated by $u_{h}^{(j)}$
\begin{equation}
  u_{h}^{(j)}(\vect{x})=\sum_{i=1}^{M}u_{i}^{(j)}\ell_{i}^{(j)}(\vect{x}),
\end{equation}
where $\ell_{i}^{(j)}\in\mathbb{V}_{(j)}^{k}$.
We denote the vector of expansion coefficients in the fine elements by $\vect{u}^{(j)}=(u_{1}^{(j)},\ldots,u_{M}^{(j)})^{\trans}$ and the basis by $\vect{\ell}^{(j)}(\vect{x})=(\ell_{1}^{(j)}(\vect{x}),\ldots,\ell_{N}^{(j)}(\vect{x}))^{\trans}$.

\section{Refinement}

Upon mesh refinement, the representation of the variable $U$ on the fine element $\vect{k}_{(j)}$ is obtained from solving the minimization problem
\begin{equation}
  \min_{\vect{u}^{(j)}}\int_{\vect{k}_{(j)}}\big(\,U_{h}(\vect{x})-u_{h}^{(j)}(\vect{x})\,\big)^{2}\,d\vect{x}.
\end{equation}
This gives
\begin{equation}
  \int_{\vect{k}_{(j)}}u_{h}^{(j)}(\vect{x})\,\ell_{i}^{(j)}(\vect{x})\,d\vect{x}
  =\int_{\vect{k}_{(j)}}U_{h}(\vect{x})\,\ell_{i}^{(j)}(\vect{x})\,d\vect{x},\quad i=1,\ldots,M.
  \label{eq:refinementProjection}
\end{equation}
Evaluating the left-hand side of Eq.~\eqref{eq:refinementProjection} gives
\begin{equation}
  \int_{\vect{k}_{(j)}}u_{h}^{(j)}(\vect{x})\,\ell_{i}^{(j)}(\vect{x})\,d\vect{x}
  =|\vect{k}_{(j)}|\,w_{i}\,u_{i}^{(j)}.
\end{equation}
\ee{Need to elaborate on this result.}
Evaluating the right-hand side of Eq.~\eqref{eq:refinementProjection} gives
\begin{equation}
  \int_{\vect{k}_{(j)}}U_{h}(\vect{x})\,\ell_{i}^{(j)}(\vect{x})\,d\vect{x}
  =|\vect{k}_{(j)}|\sum_{k=1}^{M}\int_{\vect{i}_{(j)}}\ell_{i}^{(j)}(\vect{\eta})\,\ell_{k}(\vect{\xi}(\vect{\eta}))\,d\vect{\eta}\,U_{k}.
\end{equation}
Defining the projection matrix $\mathcal{P}^{(j)}$, with components
\begin{equation}
  \mathcal{P}_{ik}^{(j)} = \int_{\vect{i}_{(j)}}\ell_{i}^{(j)}(\vect{\eta})\,\ell_{k}(\vect{\xi}(\vect{\eta}))\,d\vect{\eta},
  \label{eq:refinementProjectionMatrix}
\end{equation}
and the diagonal matrix $\mathcal{D}=\mbox{diag}(w_{1},\ldots,w_{M})$, the components of the expansion in the refined element $\vect{k}_{(j)}$ are obtained from the matrix-vector product
\begin{equation}
  \mathcal{D}\,\vect{u}^{(j)} = \mathcal{P}^{(j)}\,\vect{U}.
\end{equation}
The projection matrix can be computed once and stored at program startup.

\section{Coarsening}

In $\vect{K}$, using the fine representations, we have
\begin{equation}
  u_{h}(\vect{x}) = \sum_{j=1}^{N}\chi(\vect{k}_{(j)})\,u_{h}^{(j)}(\vect{x}),
\end{equation}
where $\chi$ is the indicator function.
Thus, upon coarsening, the representation of the variable $U$ in the coarse element is obtained by solving the minimization problem
\begin{equation}
  \min_{\vect{U}}\int_{\vect{K}}\big(\,U_{h}(\vect{x})-u_{h}(\vect{x})\,\big)^{2}\,d\vect{x}.
\end{equation}
This gives
\begin{equation}
  \int_{\vect{K}}U_{h}(\vect{x})\,\ell_{i}(\vect{x})\,d\vect{x}
  =\int_{\vect{K}}u_{h}(\vect{x})\,\ell_{i}(\vect{x})\,d\vect{x}.
  \label{eq:coarseningProjection}
\end{equation}
Evaluating the left-hand side of Eq.~\eqref{eq:coarseningProjection} gives
\begin{equation}
  \int_{\vect{K}}U_{h}(\vect{x})\,\ell_{i}(\vect{x})\,d\vect{x}
  =|\vect{K}|\,w_{i}\,U_{i}.
\end{equation}
Evaluating the right-hand side of Eq.~\eqref{eq:coarseningProjection} gives
\begin{equation}
  \int_{\vect{K}}u_{h}(\vect{x})\,\ell_{i}(\vect{x})\,d\vect{x}
  =\sum_{j=1}^{N}|\vect{k}_{(j)}|\sum_{k=1}^{M}\int_{\vect{i}_{(j)}}\ell_{i}(\vect{\xi}(\vect{\eta}))\,\ell_{k}(\vect{\eta})\,d\vect{\eta}\,u_{k}^{(j)}.
\end{equation}
Thus,
\begin{equation}
  \mathcal{D}\,\vect{U}=\sum_{j=1}^{N}\f{|\vect{k}_{(j)}|}{|\vect{K}|}
  \big(\mathcal{P}^{(j)}\big)^{\trans}\,\vect{u}^{(j)}.
\end{equation}
For a refinement factor of $2$, $|\vect{k}_{(j)}|/|\vect{K}|=2^{-d}$.

\section{Curvilinear Coordinates}

\subsection{Preliminaries}

The introduction of curvilinear coordinates doesn't change any of the preliminaries.

\subsection{Refinement}

As with Cartesian coordinates, the representation of the variable $U$ on the fine element
$\vect{k}_{\left(j\right)}$ is obtained from solving the minimization problem,
\begin{equation}
  \min_{\vect{u}^{(j)}}\int_{\vect{k}_{(j)}}\big(\,U_{h}(\vect{x})-u_{h}^{(j)}(\vect{x})\,\big)^{2}\,
  \sqrt{\gamma_{h}}\,d\vect{x},
\end{equation}
where $\sqrt{\gamma_{h}}$ is the numerical approximation to the metric determinant $\sqrt{\gamma}$.

When demanding equality of the cell averages, the metric determinant
must be taken into account, i.e., Eq.~\eqref{eq:refinementProjection} becomes
\begin{equation}
  \int_{\vect{k}_{(j)}}u_{h}^{(j)}(\vect{x})\,\ell_{i}^{(j)}(\vect{x})\,\sqrt{\gamma_{h}}\,d\vect{x}
  =\int_{\vect{k}_{(j)}}U_{h}(\vect{x})\,\ell_{i}^{(j)}(\vect{x})\,\sqrt{\gamma_{h}}\,d\vect{x},\quad i=1,\ldots,M.
  \label{eq:refinementProjection_CC}
\end{equation}
Evaluating the left-hand side of Eq.~\eqref{eq:refinementProjection_CC} gives
\begin{equation}
  \int_{\vect{k}_{(j)}}u_{h}^{(j)}(\vect{x})\,\ell_{i}^{(j)}(\vect{x})\,\sqrt{\gamma_{h}}\,d\vect{x}
  =|\vect{k}_{(j)}|\,w_{i}\,u_{i}^{(j)}\,\sqrt{\gamma_{i}}.
  \end{equation}
  Evaluating the right-hand side of Eq.~\eqref{eq:refinementProjection_CC} gives
  \begin{equation}
  \int_{\vect{k}_{(j)}}U_{h}(\vect{x})\,\ell_{i}^{(j)}(\vect{x})\,\sqrt{\gamma_{h}}\,d\vect{x}
  =|\vect{k}_{(j)}|\sum_{k=1}^{M}\int_{\vect{i}_{(j)}}\ell_{i}^{(j)}(\vect{\eta})\,\ell_{k}(\vect{\xi}(\vect{\eta}))\,\sqrt{\gamma_{h}}\,d\vect{\eta}\,U_{k}.
\end{equation}
Defining the projection matrix, $P^{\left(j\right)}_{\gamma_{h}}$, with components
\begin{equation}
P^{\left(j\right)}_{ik,\gamma_{h}}
=\int_{\vect{i}_{(j)}}\ell_{i}^{(j)}(\vect{\eta})\,\ell_{k}(\vect{\xi}(\vect{\eta}))\,\sqrt{\gamma_{h}}\,d\vect{\eta},
\end{equation}
and the diagonal matrix $\mathcal{D}_{\gamma_{h}}
=\mbox{diag}(\sqrt{\gamma_{1}}\,w_{1},\ldots,\sqrt{\gamma_{M}}\,w_{M})$, the components of the expansion in the refined element $\vect{k}_{(j)}$ are obtained from the matrix-vector product
\begin{equation}
  \mathcal{D}_{\gamma_{h}}\,\vect{u}^{(j)} = \mathcal{P}^{(j)}_{\gamma_{h}}\,\vect{U}.
\end{equation}
Since $\gamma_{h}=\gamma_{h}\left(t\right)$, $\mathcal{P}^{\left(j\right)}_{\gamma_{h}}$
and $\cD_{\gamma_{h}}$
must be computed at each time step.

\subsection{Coarsening}

The minimization problem in this case is
\begin{equation}
  \min_{\vect{U}}\int_{\vect{K}}\big(\,U_{h}(\vect{x})-u_{h}(\vect{x})\,\big)^{2}\,\sqrt{\gamma_{h}}\,d\vect{x}.
\end{equation}
As with refinement, Eq.~\eqref{eq:coarseningProjection} becomes
\begin{equation}
  \int_{\vect{K}}U_{h}(\vect{x})\,\ell_{i}(\vect{x})\,\sqrt{\gamma_{h}}\,d\vect{x}
  =\int_{\vect{K}}u_{h}(\vect{x})\,\ell_{i}(\vect{x})\,\sqrt{\gamma_{h}}\,d\vect{x}.
  \label{eq:coarseningProjection_CC}
\end{equation}
Evaluating the left-hand side of Eq.~\eqref{eq:coarseningProjection_CC} gives
\begin{equation}
  \int_{\vect{K}}U_{h}(\vect{x})\,\ell_{i}(\vect{x})\,\sqrt{\gamma_{h}}\,d\vect{x}
  =|\vect{K}|\,w_{i}\,U_{i}\,\sqrt{\gamma_{i}}.
\end{equation}
Evaluating the right-hand side of Eq.~\eqref{eq:coarseningProjection_CC} gives
\begin{equation}
  \int_{\vect{K}}u_{h}(\vect{x})\,\ell_{i}(\vect{x})\,\sqrt{\gamma_{h}}\,d\vect{x}
  =\sum_{j=1}^{N}|\vect{k}_{(j)}|\sum_{k=1}^{M}\int_{\vect{i}_{(j)}}\ell_{i}(\vect{\xi}(\vect{\eta}))\,\ell_{k}^{\left(j\right)}(\vect{\eta})\,\sqrt{\gamma_{h}}\,d\vect{\eta}\,u_{k}^{(j)}.
\end{equation}
Thus,
\begin{equation}
  \mathcal{D}_{\gamma_{h}}\,\vect{U}=\sum_{j=1}^{N}\f{|\vect{k}_{(j)}|}{|\vect{K}|}
  \big(\mathcal{P}_{\gamma_{h}}^{(j)}\big)^{\trans}\,\vect{u}^{(j)}.
\end{equation}
For a refinement factor of $2$, $|\vect{k}_{(j)}|/|\vect{K}|=2^{-d}$.

\subsection{Mapping $\eta\rightarrow\xi$}
For a 1D problem with two subcells, the mapping from the coarse reference element, coordinatized
by $\eta$, to the fine reference element, coordinatized by $\xi$, is
\begin{equation}
\xi\left(\eta\right)=\frac{1}{2}\left(\eta+\frac{1}{2}\left(-1\right)^{k}\right),\hspace{1em}k=1,2.
\end{equation}

\bibliographystyle{plain}
\bibliography{../../References/references.bib}

\end{document}
