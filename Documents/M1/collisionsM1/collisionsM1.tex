\documentclass[12pt]{article}

\usepackage{amsfonts}
\usepackage{amsmath}
\usepackage{amssymb}
\usepackage{amsthm}
\usepackage{booktabs}
\usepackage{mathrsfs}
\usepackage{graphicx}
\usepackage{cite}
\usepackage{times}
\usepackage{url}
\usepackage{hyperref}
\usepackage{lineno}
\usepackage{yhmath}
\usepackage{natbib}
\usepackage{../../definitions}
\hypersetup{
  bookmarksnumbered = true,
  bookmarksopen=false,
  pdfborder=0 0 0,         % make all links invisible, so the pdf looks good when printed
  pdffitwindow=true,      % window fit to page when opened
  pdfnewwindow=true, % links in new window
  colorlinks=true,           % false: boxed links; true: colored links
  linkcolor=blue,            % color of internal links
  citecolor=magenta,    % color of links to bibliography
  filecolor=magenta,     % color of file links
  urlcolor=cyan              % color of external links
}

\newcommand{\EC}{\mbox{\tiny{\sc Ec}}}
\newcommand{\IN}{\mbox{\tiny{\sc In}}}
\newcommand{\OUT}{\mbox{\tiny{\sc Out}}}
\newcommand{\NNS}{\mbox{\tiny NNS}} % Neutrino-Nucleon/Nuclei Scattering
\newcommand{\NES}{\mbox{\tiny{\sc Nes}}} % Neutrino-Electron Scattering
\newcommand{\PROD}{\mbox{\tiny{\sc Pr}}}
\newcommand{\ANN}{\mbox{\tiny{\sc An}}}
\newcommand{\SC}{\mbox{\tiny{\sc Sc}}}      % Scattering
\newcommand{\TP}{\mbox{\tiny{\sc Tp}}}      % Pair-Processes

\newtheorem{define}{Definition}
\newtheorem{lemma}{Lemma}
\newtheorem{prop}{Proposition}
\newtheorem{rem}{Remark}
\newtheorem{theorem}{Theorem}

\begin{document}

\title{A Spectral Two-Moment Model \\ for Neutrino-Matter Interactions}
\author{Eirik Endeve et al.}

\maketitle

\begin{abstract}
  We derive moment equations from the space homogeneous Boltzmann equation for neutrino transport and develop algorithms to solve the neutrino-matter coupling problem.  
  Electron-type neutrinos and antineutrinos are coupled to a fluid through emission and absorption, scattering, and pair processes.  
\end{abstract}

\tableofcontents

\section{Mathematical Model}

\subsection{Boltzmann Equation and Neutrino-Matter Interactions}

We denote the neutrino three-momentum with $\vect{p}=\epsilonNu\,\vect{\ell}(\omegaNu)$, the momentum space volume element with $d\vect{p}=dV_{\epsilonNu}\,d\omegaNu$, where $dV_{\epsilonNu}=\epsilonNu^{2}d\epsilonNu$ is the spherical shell energy volume element, and $d\omega$ is the momentum space angular element.  
The distribution function $f_{s}$ gives the number of neutrinos per phase-space volume
\begin{equation}
  dN_{s}=f_{s}(\vect{x},\vect{p},t)\,d\vect{x}\,d\vect{p}
  =f_{s}(\vect{x},\omegaNu,\epsilonNu,t)\,d\vect{x}\,d\omegaNu\,dV_{\epsilonNu},
\end{equation}
and the equation describing the evolution of neutrino flavor $s$ is given by (e.g., \citep{bruenn_1985})
\begin{align}
  \deriv{f_{s}}{t}(\vect{p})
  & = \big(1-f_{s}(\vect{p})\big)\,\eta_{s}(\vect{p}) - \chi_{s}(\vect{p})\,f_{s}(\vect{p}) \nonumber \\
  & \hspace{12pt}
  +\big(1-f_{s}(\vect{p})\big)\int_{\bbR^{3}}R_{s}^{\IN}(\vect{p}',\vect{p})\,f_{s}(\vect{p}')\,d\vect{p}' \nonumber \\
  & \hspace{12pt}
  -f_{s}(\vect{p})\int_{\bbR^{3}}R_{s}^{\OUT}(\vect{p}',\vect{p})\,\big(1-f_{s}(\vect{p}')\big)\,d\vect{p}' \nonumber \\
  & \hspace{12pt}
  +\big(1-f_{s}(\vect{p})\big)\int_{\bbR^{3}}R_{s}^{\PROD}(\vect{p}',\vect{p})\,\big(1-\bar{f}_{s}(\vect{p}')\big)\,d\vect{p}' \nonumber \\
  & \hspace{12pt}
  -f_{s}(\vect{p})\int_{\bbR^{3}}R_{s}^{\ANN}(\vect{p}',\vect{p})\,\bar{f}_{s}(\vect{p}')\,d\vect{p}',
  \label{eq:homogeneousBoltzmann}
\end{align}
where $\eta_{s}$ is the emissivity, $\chi_{s}$ the absorption opacity, $R_{s}^{\IN}$ and $R_{s}^{\OUT}$ scattering rates in and out of the momentum space bin centered on $\vect{p}$, and $R_{s}^{\PROD}$ and $R_{s}^{\ANN}$ are thermal production and annihilation rates due to pair processes.  
(We have adopted units where the speed of light and the Planck constant are unity.)  
In the expressions for pair processes, $\bar{f}_{s}$ is the distribution function for antineutrinos of flavor $s$.  

For thermal emission and absorption we define
\begin{equation}
  \tilde{\chi}_{s}=\eta_{s}+\chi_{s},
\end{equation}
and rewrite
\begin{equation}
  \big(1-f_{s}\big)\,\eta_{s}-\chi_{s}\,f_{s}
  =\tilde{\chi}_{s}\,\big(f_{0,s}-f_{s}\big),
\end{equation}
where the equilibrium distribution is
\begin{equation}
  f_{0,s}=\f{\eta_{s}}{\eta_{s}+\chi_{s}}.  
\end{equation}
In beta equilibrium, $f_{0,s}$ is given by the Fermi-Dirac distribution
\begin{equation}
  f_{0,s}=\f{1}{e^{(\varepsilon-\mu_{s})/kT}+1},
\end{equation}
where $\mu_{s}$ is the neutrino chemical potential and $T$ is the matter temperature.  
For electron neutrinos ($s=\nu_{e}$), $\mu_{\nu_{e}}=\mu_{e^{-}}+(\mu_{p}-\mu_{n})$, while for electron antineutrinos ($s=\bar{\nu}_{e}$), $\mu_{\bar{\nu}_{e}}=\mu_{e^{+}}-(\mu_{p}-\mu_{n})$.  
Here, $\mu_{e^{-(+)}}$ is the electron (positron) chemical potential, $\mu_{n}$ the neutron chemical potential, and $\mu_{p}$ the proton chemical potential, which are evaluated with an appropriate equation of state (EoS).  

We exploit the fact that the scattering and pair process kernels depend on the energies $\varepsilon$ and $\varepsilon'$ and the cosine of the scattering angle $\cos\alpha=\vect{\ell}\cdot\vect{\ell}'$, and introduce the $L$-term Legendre expansion
\begin{equation}
  R(\vect{p},\vect{p}')
  =R(\varepsilon,\varepsilon',\vect{\ell}\cdot\vect{\ell}')
  \approx\sum_{l=0}^{L}\Phi_{l}(\varepsilon,\varepsilon')\,P_{l}(\vect{\ell}\cdot\vect{\ell}').  
\end{equation}
From orthogonality of the Legendre polynomials, we have
\begin{equation}
  \Phi_{l}(\varepsilon,\varepsilon')
  =\f{1}{C_{l}}\int_{-1}^{1}R(\varepsilon,\varepsilon',x)\,P_{l}(x)\,dx,
\end{equation}
where $C_{l}$ ($l=0,\ldots,L$) are normalization constants.  
In the following, as a first approximation, we will only include the isotropic part of the kernels (i.e., $L=0$).  

\subsection{Angular Moment Equations}

In the spectral two-moment model we solve for the zeroth and first moments of the neutrino distribution function, defined respectively as
\begin{equation}
  \big\{\,\cJ_{s},\vect{\cH}_{s}\,\big\}(\varepsilon)=\f{1}{4\pi}\int_{\bbS^{2}}f_{s}(\varepsilon,\omega)\,\big\{\,1,\vect{\ell}(\omega)\,\big\}\,d\omega,
\end{equation}
with moments for antineutrinos ($\bar{\cJ}_{s}$ and $\bar{\vect{\cH}}_{s}$) defined analogously.  
Then, the zeroth moment $\cJ_{s}$ ($\bar{\cJ}_{s}$) is the spectral number density of neutrinos (antineutrinos), and the first moment $\vect{\cH}_{s}$ ($\bar{\vect{\cH}}_{s}$) is the spectral number flux density of neutrinos (antineutrinos).  
Since the distribution function is bounded by $0\le f_{s}\le1$, it can be shown that the moments must satisfy the bounds
\begin{equation}
  0\le\cJ_{s}\le1
  \quad\text{and}\quad
  |\vect{\cH}_{s}|\le\big(1-\cJ_{s}\big)\,\cJ_{s}.  
\end{equation}
(Similar bounds hold for antineutrinos.)

Taking the zeroth moment of Eq.~\eqref{eq:homogeneousBoltzmann} results in the moment equation for the number density
\begin{equation}
  \deriv{\cJ_{s}}{t}
  =\widehat{\eta}_{s} - \widehat{\chi}_{s}\,\cJ_{s},
  \label{eq:momentNumber}
\end{equation}
where we have defined the total emissivity
\begin{equation}
  \widehat{\eta}_{s}(\cJ_{s},\bar{\cJ}_{s})
  =\tilde{\chi}_{s}\,\cJ_{0,s} + \eta_{\SC,s}(\cJ_{s}) + \eta_{\TP,s}(\bar{\cJ}_{s})
\end{equation}
and total opacity
\begin{equation}
  \widehat{\chi}_{s}(\cJ_{s},\bar{\cJ}_{s})
  =\tilde{\chi}_{s} + \chi_{\SC,s}(\cJ_{s}) + \chi_{\TP,s}(\bar{\cJ}_{s}).  
\end{equation}
We have also defined the scattering emissivity
\begin{equation}
  \eta_{\SC,s}(\cJ_{s})
  =\int_{\bbR^{+}}\Phi_{0,s}^{\IN}(\varepsilon,\varepsilon')\,\cJ_{s}(\varepsilon')\,dV_{\varepsilon'},
\end{equation}
the scattering opacity
\begin{equation}
  \chi_{\SC,s}(\cJ_{s})
  =\int_{\bbR^{+}}\big[\,\Phi_{0.s}^{\IN}(\varepsilon,\varepsilon')\,\cJ(\varepsilon')+\Phi_{0,s}^{\Out}(\varepsilon,\varepsilon')\,\big(1-\cJ(\varepsilon')\big)\,\big]\,dV_{\varepsilon'},
\end{equation}
the emissivity due to thermal pair processes
\begin{equation}
  \eta_{\TP,s}(\bar{\cJ}_{s})
  =\int_{\bbR^{+}}\Phi_{0,s}^{\PROD}(\varepsilon,\varepsilon')\,\big(1-\bar{\cJ}_{s}(\varepsilon')\big)\,dV_{\varepsilon'},
\end{equation}
and the opacity due to thermal pair processes
\begin{equation}
  \chi_{\TP,s}(\bar{\cJ}_{s})
  =\int_{\bbR^{+}}\big[\,\Phi_{0,s}^{\PROD}(\varepsilon,\varepsilon')\,\big(1-\bar{\cJ}_{s}(\varepsilon')\big)+\Phi_{0,s}^{\ANN}(\varepsilon,\varepsilon')\,\bar{\cJ}_{s}(\varepsilon')\,\big]\,dV_{\varepsilon'}.  
\end{equation}
Note that the opacities due to scattering and pair processes depend on $\cJ_{s}$ and $\bar{\cJ}_{s}$, respectively, due to the Fermi blocking factors.  
Since $0\le\cJ_{s},\bar{\cJ}_{s}\le1$, we have $\eta_{\SC,s},\chi_{\SC,s},\eta_{\TP,s},\chi_{\TP,s}\ge0$.  
Also note that the emissivities and opacities depend on the neutrino energy $\varepsilon$ and local thermodynamic conditions (e.g., density $\rho$, temperature $T$, and electron fraction $Y_{e}$).  

Similarly, taking the first moment of Eq.~\eqref{eq:homogeneousBoltzmann} results in the equation for the number flux density
\begin{equation}
  \deriv{\vect{\cH}_{s}}{t}
  =-\widehat{\chi}_{s}\,\vect{\cH}_{s}.
  \label{eq:momentFlux}
\end{equation}

\subsection{Matter Equations}

Neutrino-matter interactions mediate exchange of lepton number, momentum, and energy between matter and neutrinos.  
As a first approximation, we will assume that the fluid remains static and ignore momentum exchange.  
Only emission and absorption due to electron capture modify $Y_{e}$, while all processes modify the internal energy.  
The electron fraction $Y_{e}$ and internal energy $e=\rho\,\epsilon$ evolve according to
\begin{equation}
  \rho\deriv{Y_{e}}{t}
  =-m_{b}\int_{\bbR^{+}}
  \big[\,
    \tilde{\chi}_{\nu_{e}}\,\big(\cJ_{0,\nu_{e}}-\cJ_{\nu_{e}}\,\big)
    -\tilde{\chi}_{\bar{\nu}_{e}}\,\big(\,\cJ_{0,\bar{\nu}_{e}}-\cJ_{\bar{\nu}_{e}}\,\big)
  \,\big]\,dV_{\varepsilon},
\end{equation}
and
\begin{equation}
  \rho\deriv{\epsilon}{t}
  =-\sum_{s}\int_{\bbR^{+}}\big(\,\widehat{\eta}_{s} - \widehat{\chi}_{s}\,\cJ_{s}\,\big)\,\varepsilon\,dV_{\varepsilon},
\end{equation}
where $\rho$ is the mass density and $m_{b}$ is the average baryon mass.  

The matter sources are constructed to ensure conservation of lepton number and energy.  
The neutrino density $J_{s}(\vect{x},t)$ and energy $E_{s}(\vect{x},t)$ are defined as
\begin{equation}
  \big\{\,J_{s},\,E_{s}\,\big\}
  =\int_{\bbR^{+}}\cJ_{s}(\varepsilon)\big\{\,1,\,\varepsilon\,\big\}\,dV_{\varepsilon}.
\end{equation}
Lepton number conservation is due to
\begin{equation}
  \deriv{}{t}\big(\,J_{\nu_{e}}-J_{\bar{\nu}_{e}}\,\big)+\f{\rho}{m_{b}}\deriv{Y_{e}}{t} = 0,
\end{equation}
while energy conservation is due to
\begin{equation}
  \sum_{s}\deriv{E_{s}}{t}+\rho\deriv{\epsilon}{t} = 0.
\end{equation}

\section{Implicit Solvers}

We discuss the implicit solver in the context of a system of electron neutrinos and antineutrinos.  
We let $\mathcal{J}_{\nu_{e}}\to\mathcal{J}$ and $\mathcal{J}_{\bar{\nu}_{e}}\to\bar{\mathcal{J}}$.  
The electron fraction is updated from
\begin{equation}
  \f{\rho}{m_{b}}\big(Y_{e}^{n+1}-Y_{e}^{n}\big) + \big[\,\big(J^{n+1}-\bar{J}^{n+1}\big)-\big(J^{n}-\bar{J}^{n}\big)\,\big] = 0,
  \label{eq:electronFractionContinuous}
\end{equation}
where
\begin{equation}
  \big\{\,J,\bar{J}\,\big\}
  =\sum_{q=1}^{N_{\varepsilon}}\hat{w}_{q}^{(2)}\,\big\{\,\mathcal{J}_{q},\bar{\mathcal{J}}_{q}\,\big\}
  \quad\text{with}\quad
  \hat{w}_{q}^{(2)}=\f{4\pi}{h^{3}}\,w_{q}^{(2)},
\end{equation}
where $w_{q}^{(2)}$ are quadrature weights for energy integrals with respect to $dV_{\varepsilon}=\varepsilon^{2}\,d\varepsilon$.  
Defining $S_{Y_{e}}\equiv \rho\,Y_{e}^{n}/m_{b}$ and $U_{Y_{e}} \equiv Y_{e}/Y_{e}^{n}$, we can write Eq.~\eqref{eq:electronFractionContinuous} in discrete form as
\begin{equation}
  F_{Y_{e}}(\vect{U}^{n+1})\equiv
  U_{Y_{e}}^{n+1} + \sum_{q}\hat{w}_{q}^{(2)}\big(\,\mathcal{J}_{q}^{n+1}-\bar{\mathcal{J}}_{q}^{n+1}\,\big)/S_{Y_{e}} 
  - \big\{\,1+\sum_{q}\hat{w}_{q}^{(2)}\big(\,\mathcal{J}_{q}^{n}-\bar{\mathcal{J}}_{q}^{n}\,\big)/S_{Y_{e}}\,\big\} = 0.
  \label{eq:electronFractionDiscrete}
\end{equation}
Here we have defined the solution vector $\vect{U}=\big(Y_{e},\epsilon,\vect{\mathcal{J}},\vect{\bar{\mathcal{J}}}\big)^{T}$, where $\vect{\mathcal{J}}=\big(\mathcal{J}_{1},\ldots,\mathcal{J}_{N_{\varepsilon}}\big)^{T}$ and $\vect{\bar{\mathcal{J}}}=\big(\bar{\mathcal{J}}_{1},\ldots,\bar{\mathcal{J}}_{N_{\varepsilon}}\big)^{T}$.  
We also define the vectors $\vect{u}=\big(Y_{e},\epsilon\big)^{T}$ and $\vect{\mathcal{U}}=\big(\vect{\mathcal{J}},\vect{\bar{\mathcal{J}}}\big)^{T}$.  

The specific internal energy is updated from
\begin{equation}
  \rho\,\big(\epsilon^{n+1}-\epsilon^{n}\big)
  +\big[\,\big(E^{n+1}+\bar{E}^{n+1}\big)-\big(E^{n}+\bar{E}^{n}\big)\,\big] = 0,
  \label{eq:internalEnergyContinuous}
\end{equation}
where
\begin{equation}
  \big\{\,E,\bar{E}\,\big\}
  =\sum_{q=1}^{N_{\varepsilon}}\hat{w}_{q}^{(3)}\,\big\{\,\mathcal{J}_{q},\bar{\mathcal{J}}_{q}\,\big\}
  \quad\text{with}\quad
  \hat{w}_{q}^{(3)}=\f{4\pi}{h^{3}}\,w_{q}^{(3)},
\end{equation}
where $w_{q}^{(3)}$ are quadrature weights for energy integrals with respect to $\varepsilon\,dV_{\varepsilon}=\varepsilon^{3}\,d\varepsilon$.  
Defining $S_{\epsilon}\equiv\rho\epsilon^{n}$ and $U_{\epsilon}\equiv\epsilon/\epsilon^{n}$, we can write Eq.~\eqref{eq:internalEnergyContinuous} as
\begin{equation}
  F_{\epsilon}(\vect{U}^{n+1})\equiv
  U_{\epsilon}^{n+1} + \sum_{q}\hat{w}_{q}^{(3)}\big(\,\mathcal{J}_{q}^{n+1}+\bar{\mathcal{J}}_{q}^{n+1}\,\big)/S_{\epsilon}
  -\big\{\,1+\sum_{q}\hat{w}_{q}^{(3)}\big(\,\mathcal{J}_{q}^{n}+\bar{\mathcal{J}}_{q}^{n}\,\big)/S_{\epsilon}\,\big\} = 0.  
  \label{eq:internalEnergyDiscrete}
\end{equation}

The neutrino density is updated from
\begin{align}
  \f{\big(\mathcal{J}_{i}^{n+1}-\mathcal{J}_{i}^{n}\big)}{\dt}
  &=\big[\chi_{\EC}\big]_{i}^{n}\,\big(\mathcal{J}_{0,i}^{n+1}-\mathcal{J}_{i}^{n+1}\big) \nonumber \\
  &\hspace{12pt}
  +\sum_{q}w_{q}^{(2)}\,\big[\Phi_{\NES,0}^{\In}\big]_{qi}^{n+1}\,\mathcal{J}_{q}^{n+1} \nonumber \\
  &\hspace{12pt}
  -\mathcal{J}_{i}^{n+1}\sum_{q}w_{q}^{(2)}\,\big\{\,\big[\Phi_{\NES,0}^{\In}\big]_{qi}^{n+1}\,\mathcal{J}_{q}^{n+1}+\big[\Phi_{\NES,0}^{\Out}\big]_{qi}^{n+1}\,\big(1-\mathcal{J}_{q}^{n+1}\big)\,\big\} \nonumber \\
  &\hspace{12pt}
  + \sum_{q}w_{q}^{(2)}\,\big[\Phi_{\TP,0}^{\PROD}\big]_{qi}^{n+1}\,\big(1-\bar{\mathcal{J}}_{q}^{n+1}\big) \nonumber \\
  &\hspace{12pt}
  -\mathcal{J}_{i}^{n+1}\sum_{q}w_{q}^{(2)}\big\{\,\big[\Phi_{\TP,0}^{\PROD}\big]_{qi}^{n+1}\,\big(1-\bar{\mathcal{J}}_{q}^{n+1}\big)+\big[\Phi_{\TP,0}^{\ANN}\big]_{qi}^{n+1}\,\bar{\mathcal{J}}_{q}^{n+1}\,\big\} \nonumber \\
  &=\big(\big[\eta_{\EC}\big]_{i}^{n+1}+\big[\eta_{\NES}\big]_{i}^{n+1}+\big[\eta_{\TP}\big]_{i}^{n+1}\big) \nonumber \\
  &\hspace{12pt}
  -\big(\big[\chi_{\EC}\big]_{i}^{n}+\big[\chi_{\NES}\big]_{i}^{n+1}+\big[\chi_{\TP}\big]_{i}^{n+1}\big)\,\mathcal{J}_{i}^{n+1} \nonumber \\
  &=\big[\eta\big]_{i}^{n+1} - \big[\chi\big]_{i}^{n+1}\,\mathcal{J}_{i}^{n+1},
  \label{eq:neutrinoDiscrete}
\end{align}
where we have defined
\begin{align}
  \big[\eta_{\EC}\big]_{i}^{n+1}
  &=\big[\chi_{\EC}\big]_{i}^{n}\,\mathcal{J}_{0,i}^{n+1}, \label{eq:ECemissivityNeutrino} \\
  \big[\eta_{\NES}\big]_{i}^{n+1}
  &=\sum_{q}w_{q}^{(2)}\,\big[\Phi_{\NES,0}^{\In}\big]_{qi}^{n+1}\,\mathcal{J}_{q}^{n+1}, \label{eq:NESemissivityNeutrino} \\
  \big[\eta_{\TP}\big]_{i}^{n+1}
  &=\sum_{q}w_{q}^{(2)}\,\big[\Phi_{\TP,0}^{\PROD}\big]_{qi}^{n+1}\,\big(1-\bar{\mathcal{J}}_{q}^{n+1}\big), \label{eq:PAIRemissivityNeutrino} \\
  \big[\chi_{\NES}\big]_{i}^{n+1}
  &=\sum_{q}w_{q}^{(2)}\,\big\{\,\big[\Phi_{\NES,0}^{\In}\big]_{qi}^{n+1}\,\mathcal{J}_{q}^{n+1}+\big[\Phi_{\NES,0}^{\Out}\big]_{qi}^{n+1}\,\big(1-\mathcal{J}_{q}^{n+1}\big)\,\big\} \\
  &=\big[\eta_{\NES}\big]_{i}^{n+1}+\sum_{q}w_{q}^{(2)}\,\big[\Phi_{\NES,0}^{\Out}\big]_{qi}^{n+1}\,\big(1-\mathcal{J}_{q}^{n+1}\big), \label{eq:NESopacityNeutrino} \\
  \big[\chi_{\TP}\big]_{i}^{n+1}
  &=\sum_{q}w_{q}^{(2)}\big\{\,\big[\Phi_{\TP,0}^{\PROD}\big]_{qi}^{n+1}\,\big(1-\bar{\mathcal{J}}_{q}^{n+1}\big)+\big[\Phi_{\TP,0}^{\ANN}\big]_{qi}^{n+1}\,\bar{\mathcal{J}}_{q}^{n+1}\,\big\} \\
  &=\big[\eta_{\TP}\big]_{i}^{n+1}+\sum_{q}w_{q}^{(2)}\,\big[\Phi_{\TP,0}^{\ANN}\big]_{qi}^{n+1}\,\bar{\mathcal{J}}_{q}^{n+1}, \label{eq:PAIRopacityNeutrino} \\
  \big[\eta\big]_{i}^{n+1}
  &=\big[\eta_{\EC}\big]_{i}^{n+1}+\big[\eta_{\NES}\big]_{i}^{n+1}+\big[\eta_{\TP}\big]_{i}^{n+1}, \label{eq:TOTALemissivityNeutrino} \\
  \big[\chi\big]_{i}^{n+1}
  &=\big[\chi_{\EC}\big]_{i}^{n}+\big[\chi_{\NES}\big]_{i}^{n+1}+\big[\chi_{\TP}\big]_{i}^{n+1}.  \label{eq:TOTALopacityNeutrino}
\end{align}
We rewrite Eq.~\eqref{eq:neutrinoDiscrete} and define
\begin{equation}
  \mathcal{F}_{i}(\vect{U}^{n+1})
  \equiv\big(\,1+\dt\,\big[\chi\big]_{i}^{n+1}\,\big)\,\mathcal{J}_{i}^{n+1}-\big(\,\mathcal{J}_{i}^{n}+\dt\,\big[\eta\big]_{i}^{n+1}\,\big)=0.
  \label{eq:neutrinoDiscreteResidualForm}
\end{equation}
Similarly, the antineutrino density is updated from
\begin{align}
  \f{\big(\bar{\mathcal{J}}_{i}^{n+1}-\bar{\mathcal{J}}_{i}^{n}\big)}{\dt}
  &=\big[\bar{\chi}_{\EC}\big]_{i}^{n}\,\big(\bar{\mathcal{J}}_{0,i}^{n+1}-\bar{\mathcal{J}}_{i}^{n+1}\big) \nonumber \\
  &\hspace{12pt}
  +\sum_{q}w_{q}^{(2)}\,\big[\bar{\Phi}_{\NES,0}^{\In}\big]_{qi}^{n+1}\,\bar{\mathcal{J}}_{q}^{n+1} \nonumber \\
  &\hspace{12pt}
  -\bar{\mathcal{J}}_{i}^{n+1}\sum_{q}w_{q}^{(2)}\,\big\{\,\big[\bar{\Phi}_{\NES,0}^{\In}\big]_{qi}^{n+1}\,\bar{\mathcal{J}}_{q}^{n+1}+\big[\bar{\Phi}_{\NES,0}^{\Out}\big]_{qi}^{n+1}\,\big(1-\bar{\mathcal{J}}_{q}^{n+1}\big)\,\big\} \nonumber \\
  &\hspace{12pt}
  + \sum_{q}w_{q}^{(2)}\,\big[\bar{\Phi}_{\TP,0}^{\PROD}\big]_{qi}^{n+1}\,\big(1-\mathcal{J}_{q}^{n+1}\big) \nonumber \\
  &\hspace{12pt}
  -\bar{\mathcal{J}}_{i}^{n+1}\sum_{q}w_{q}^{(2)}\big\{\,\big[\bar{\Phi}_{\TP,0}^{\PROD}\big]_{qi}^{n+1}\,\big(1-\mathcal{J}_{q}^{n+1}\big)+\big[\bar{\Phi}_{\TP,0}^{\ANN}\big]_{qi}^{n+1}\,\mathcal{J}_{q}^{n+1}\,\big\} \nonumber \\
  &=\big(\,\big[\bar{\eta}_{\EC}\big]_{i}^{n+1}+\big[\bar{\eta}_{\NES}\big]_{i}^{n+1}+\big[\bar{\eta}_{\TP}\big]_{i}^{n+1}\,\big) \nonumber \\
  &\hspace{12pt}
  -\big(\,\big[\bar{\chi}_{\EC}\big]_{i}^{n}+\big[\bar{\chi}_{\NES}\big]_{i}^{n+1}+\big[\bar{\chi}_{\TP}\big]_{i}^{n+1}\,\big)\,\bar{\mathcal{J}}_{i}^{n+1} \nonumber \\
  &=\big[\bar{\eta}\big]_{i}^{n+1} - \big[\bar{\chi}\big]_{i}^{n+1}\,\bar{\mathcal{J}}_{i}^{n+1},
  \label{eq:antineutrinoDiscrete}
\end{align}
where we have defined
\begin{align}
  \big[\bar{\eta}_{\EC}\big]_{i}^{n+1}
  &=\big[\bar{\chi}_{\EC}\big]_{i}^{n}\,\bar{\mathcal{J}}_{0,i}^{n+1}, \label{eq:ECemissivityAntineutrino} \\
  \big[\bar{\eta}_{\NES}\big]_{i}^{n+1}
  &=\sum_{q}w_{q}^{(2)}\,\big[\bar{\Phi}_{\NES,0}^{\In}\big]_{qi}^{n+1}\,\bar{\mathcal{J}}_{q}^{n+1}, \label{eq:NESemissivityAntineutrino} \\
  \big[\bar{\eta}_{\TP}\big]_{i}^{n+1}
  &=\sum_{q}w_{q}^{(2)}\,\big[\bar{\Phi}_{\TP,0}^{\PROD}\big]_{qi}^{n+1}\,\big(1-\mathcal{J}_{q}^{n+1}\big), \label{eq:PAIRemissivityAntineutrino} \\
  \big[\bar{\chi}_{\NES}\big]_{i}^{n+1}
  &=\sum_{q}w_{q}^{(2)}\,\big\{\,\big[\bar{\Phi}_{\NES,0}^{\In}\big]_{qi}^{n+1}\,\bar{\mathcal{J}}_{q}^{n+1}+\big[\bar{\Phi}_{\NES,0}^{\Out}\big]_{qi}^{n+1}\,\big(1-\bar{\mathcal{J}}_{q}^{n+1}\big)\,\big\} \\
  &=\big[\bar{\eta}_{\NES}\big]_{i}^{n+1}+\sum_{q}w_{q}^{(2)}\,\big[\bar{\Phi}_{\NES,0}^{\Out}\big]_{qi}^{n+1}\,\big(1-\bar{\mathcal{J}}_{q}^{n+1}\big), \label{eq:NESopacityAntineutrino} \\
  \big[\bar{\chi}_{\TP}\big]_{i}^{n+1}
  &=\sum_{q}w_{q}^{(2)}\big\{\,\big[\bar{\Phi}_{\TP,0}^{\PROD}\big]_{qi}^{n+1}\,\big(1-\mathcal{J}_{q}^{n+1}\big)+\big[\bar{\Phi}_{\TP,0}^{\ANN}\big]_{qi}^{n+1}\,\mathcal{J}_{q}^{n+1}\,\big\} \\
  &=\big[\bar{\eta}_{\TP}\big]_{i}^{n+1}+\sum_{q}w_{q}^{(2)}\,\big[\bar{\Phi}_{\TP,0}^{\ANN}\big]_{qi}^{n+1}\,\mathcal{J}_{q}^{n+1}, \label{eq:PAIRopacityAntineutrino} \\
  \big[\bar{\eta}\big]_{i}^{n+1}
  &=\big[\bar{\eta}_{\EC}\big]_{i}^{n+1}+\big[\bar{\eta}_{\NES}\big]_{i}^{n+1}+\big[\bar{\eta}_{\TP}\big]_{i}^{n+1}, \label{eq:TOTALemissivityAntineutrino} \\
  \big[\bar{\chi}\big]_{i}^{n+1}
  &=\big[\bar{\chi}_{\EC}\big]_{i}^{n}+\big[\bar{\chi}_{\NES}\big]_{i}^{n+1}+\big[\bar{\chi}_{\TP}\big]_{i}^{n+1}. \label{eq:TOTALopacityAntineutrino}
\end{align}
We rewrite Eq.~\eqref{eq:antineutrinoDiscrete} and define
\begin{equation}
  \bar{\mathcal{F}}_{i}(\vect{U}^{n+1})\equiv
  \big(\,1+\dt\,\big[\bar{\chi}\big]_{i}^{n+1}\,\big)\,\bar{\mathcal{J}}_{i}^{n+1}-\big(\,\bar{\mathcal{J}}_{i}^{n}+\dt\,\big[\bar{\eta}\big]_{i}^{n+1}\,\big)=0.
  \label{eq:antineutrinoDiscreteResidualForm}
\end{equation}

Next, we define
\begin{align}
  \vect{\mathcal{F}}(\vect{U})=\big(\,\mathcal{F}_{1}(\vect{U}),\ldots,\mathcal{F}_{N_{\varepsilon}}(\vect{U})\,\big)^{T}, \\
  \vect{\bar{\mathcal{F}}}(\vect{U})=\big(\,\bar{\mathcal{F}}_{1}(\vect{U}),\ldots,\bar{\mathcal{F}}_{N_{\varepsilon}}(\vect{U})\,\big)^{T}
\end{align}
and
\begin{equation}
  \vect{F}(\vect{U})=\big(F_{Y_{e}}(\vect{U}),F_{\epsilon}(\vect{U}),\vect{\mathcal{F}}(\vect{U}),\vect{\bar{\mathcal{F}}}(\vect{U})\big)^{T},
\end{equation}
so that we can write the coupled nonlinear system consisting of Eqs.~\eqref{eq:electronFractionDiscrete}, \eqref{eq:internalEnergyDiscrete}, \eqref{eq:neutrinoDiscreteResidualForm}, and \eqref{eq:antineutrinoDiscreteResidualForm} in the compact form
\begin{equation}
  \vect{F}(\vect{U}) = 0.
  \label{eq:nonlinearSystemResidualForm}
\end{equation}

\subsection{Newton's Method}

Newton's method if given by
\begin{itemize}
  \item[1.] Set $\vect{U}^{[0]} = \vect{U}^{n}$
  \item[2.] For $k=1,\ldots,M$ compute
  \begin{equation}
    \vect{U}^{[k]}=\vect{U}^{[k-1]} + \delta\vect{U}^{[k-1]}
  \end{equation}
  \item[3.] Set $\vect{U}^{n+1}=\vect{U}^{[M]}$,
\end{itemize}
where
\begin{equation}
  \delta\vect{U}^{[k]} = - \vect{J}(\vect{U}^{[k]})^{-1}\,\vect{F}(\vect{U}^{[k]}), \quad\text{with}\quad \vect{J}(\vect{U}^{[k]})=\vect{F}'(\vect{U}^{[k]}),
\end{equation}
and where the number of iterations $M$ is given by suitable convergence criteria.  

From Eq.~\eqref{eq:electronFractionDiscrete}, the elements of the Jacobian matrix are
\begin{align}
  \pderiv{F_{Y_{e}}}{Y_{e}}
  &=\f{1}{Y_{e}^{n}}, \\
  \pderiv{F_{Y_{e}}}{\epsilon}
  &=0, \\
  \pderiv{F_{Y_{e}}}{\mathcal{J}_{j}}
  &=\f{\hat{w}_{j}^{(2)}}{S_{Y_{e}}}, \\
  \pderiv{F_{Y_{e}}}{\bar{\mathcal{J}}_{j}}
  &=-\f{\hat{w}_{j}^{(2)}}{S_{Y_{e}}}.  
\end{align}
From Eq.~\eqref{eq:internalEnergyDiscrete}, the elements of the Jacobian matrix are
\begin{align}
  \pderiv{F_{\epsilon}}{Y_{e}}
  &=0, \\
  \pderiv{F_{\epsilon}}{\epsilon}
  &=\f{1}{\epsilon^{n}}, \\
  \pderiv{F_{\epsilon}}{\mathcal{J}_{j}}
  &=\f{\hat{w}_{j}^{(3)}}{S_{\epsilon}}, \\
  \pderiv{F_{\epsilon}}{\bar{\mathcal{J}}_{j}}
  &=\f{\hat{w}_{j}^{(3)}}{S_{\epsilon}}
\end{align}
From Eq.~\eqref{eq:neutrinoDiscreteResidualForm}, the elements of the Jacobian are
\begin{align}
  \pderiv{\mathcal{F}_{i}}{Y_{e}}
  &=\dt\,\Big(\,\pderiv{\big[\chi\big]_{i}}{Y_{e}}\,\mathcal{J}_{i}-\pderiv{\big[\eta\big]_{i}}{Y_{e}}\,\Big), \\
  \pderiv{\mathcal{F}_{i}}{\epsilon}
  &=\dt\,\Big(\,\pderiv{\big[\chi\big]_{i}}{\epsilon}\,\mathcal{J}_{i}-\pderiv{\big[\eta\big]_{i}}{\epsilon}\,\Big), \\
  \pderiv{\mathcal{F}_{i}}{\mathcal{J}_{j}}
  &=\big(\,1+\dt\,\big[\chi\big]_{i}\,\big)\,\delta_{ij}
  +\dt\,\Big(\,\pderiv{\big[\chi\big]_{i}}{\mathcal{J}_{j}}\,\mathcal{J}_{i}-\pderiv{\big[\eta\big]_{i}}{\mathcal{J}_{j}}\,\Big), \\
  \pderiv{\mathcal{F}_{i}}{\bar{\mathcal{J}}_{j}}
  &=\dt\,\Big(\,\pderiv{\big[\chi\big]_{i}}{\bar{\mathcal{J}}_{j}}\,\mathcal{J}_{i}-\pderiv{\big[\eta\big]_{i}}{\bar{\mathcal{J}}_{j}}\,\Big).  
\end{align}
The derivatives of neutrino emissivities and opacities with respect to $Y_{e}$ are
\begin{align}
  \pderiv{\big[\eta\big]_{i}}{Y_{e}}
  &=\big[\chi_{\EC}\big]_{i}^{n}\,\pderiv{\mathcal{J}_{0,i}}{Y_{e}} \nonumber \\
  &\hspace{12pt}
  +\sum_{q}w_{q}^{(2)}\,\pderiv{\big[\Phi_{\NES,0}^{\In}\big]_{qi}}{Y_{e}}\,\mathcal{J}_{q}^{n+1}
  +\sum_{q}w_{q}^{(2)}\,\pderiv{\big[\Phi_{\TP,0}^{\PROD}\big]_{qi}}{Y_{e}}\,\big(1-\bar{\mathcal{J}}_{q}^{n+1}\big) , \\
  \pderiv{\big[\chi\big]_{i}}{Y_{e}}
  &=\sum_{q}w_{q}^{(2)}\,\Big\{\,\pderiv{\big[\Phi_{\NES,0}^{\In}\big]_{qi}}{Y_{e}}\,\mathcal{J}_{q}^{n+1}+\pderiv{\big[\Phi_{\NES,0}^{\Out}\big]_{qi}}{Y_{e}}\,\big(1-\mathcal{J}_{q}^{n+1}\big)\,\Big\} \nonumber \\
  &\hspace{12pt}
  +\sum_{q}w_{q}^{(2)}\Big\{\,\pderiv{\big[\Phi_{\TP,0}^{\PROD}\big]_{qi}}{Y_{e}}\,\big(1-\bar{\mathcal{J}}_{q}^{n+1}\big)+\pderiv{\big[\Phi_{\TP,0}^{\ANN}\big]_{qi}}{Y_{e}}\,\bar{\mathcal{J}}_{q}^{n+1}\,\Big\}.  
\end{align}
The derivatives of neutrino emissivities and opacities with respect to $\epsilon$ are
\begin{align}
  \pderiv{\big[\eta\big]_{i}}{\epsilon}
  &=\big[\chi_{\EC}\big]_{i}^{n}\,\pderiv{\mathcal{J}_{0,i}}{\epsilon} \nonumber \\
  &\hspace{12pt}
  +\sum_{q}w_{q}^{(2)}\,\pderiv{\big[\Phi_{\NES,0}^{\In}\big]_{qi}}{\epsilon}\,\mathcal{J}_{q}^{n+1}
  +\sum_{q}w_{q}^{(2)}\,\pderiv{\big[\Phi_{\TP,0}^{\PROD}\big]_{qi}}{\epsilon}\,\big(1-\bar{\mathcal{J}}_{q}^{n+1}\big) , \\
  \pderiv{\big[\chi\big]_{i}}{\epsilon}
  &=\sum_{q}w_{q}^{(2)}\,\Big\{\,\pderiv{\big[\Phi_{\NES,0}^{\In}\big]_{qi}}{\epsilon}\,\mathcal{J}_{q}^{n+1}+\pderiv{\big[\Phi_{\NES,0}^{\Out}\big]_{qi}}{\epsilon}\,\big(1-\mathcal{J}_{q}^{n+1}\big)\,\Big\} \nonumber \\
  &\hspace{12pt}
  +\sum_{q}w_{q}^{(2)}\Big\{\,\pderiv{\big[\Phi_{\TP,0}^{\PROD}\big]_{qi}}{\epsilon}\,\big(1-\bar{\mathcal{J}}_{q}^{n+1}\big)+\pderiv{\big[\Phi_{\TP,0}^{\ANN}\big]_{qi}}{\epsilon}\,\bar{\mathcal{J}}_{q}^{n+1}\,\Big\}.  
\end{align}
The derivatives of neutrino emissivities and opacities with respect to $\mathcal{J}_{j}$ are
\begin{align}
  \pderiv{\big[\eta\big]_{i}}{\mathcal{J}_{j}}
  &=w^{(2)}_{j}\,\big[\Phi_{\NES,0}^{\In}\big]_{ji}, \\
  \pderiv{\big[\chi\big]_{i}}{\mathcal{J}_{j}}
  &=w^{(2)}_{j}\,\Big(\,\big[\Phi_{\NES,0}^{\In}\big]_{ji}-\big[\Phi_{\NES,0}^{\Out}\big]_{ji}\,\Big).  
\end{align}
The derivatives of neutrino emissivities and opacities with respect to $\bar{\mathcal{J}}_{j}$ are
\begin{align}
  \pderiv{\big[\eta\big]_{i}}{\bar{\mathcal{J}}_{j}}
  &=-w_{j}^{(2)}\,\big[\Phi_{\TP,0}^{\PROD}\big]_{ji} , \\
  \pderiv{\big[\chi\big]_{i}}{\bar{\mathcal{J}}_{j}}
  &=w_{j}^{(2)}\,\big(\,\big[\Phi_{\TP,0}^{\ANN}\big]_{ji}-\big[\Phi_{\TP,0}^{\PROD}\big]_{ji}\,\big) .
\end{align}

\bibliographystyle{plain}
\bibliography{../../References/references.bib}

\appendix

\section{Kernel Derivatives}

NES and pair kernels are tabulated in terms of temperature $T$ and degeneracy parameter $\eta=\mu_{e}/kT$
\begin{equation}
  \Phi=\Phi(T,\eta).  
\end{equation}
For solvers based on Newton's method, we need derivatives with respect to internal energy $\epsilon$ and electron fraction $Y_{e}$.  
\begin{align}
  d\Phi 
  &= \Big(\pderiv{\Phi}{T}\Big)_{\eta}\,dT+\Big(\pderiv{\Phi}{\eta}\Big)_{T}\,d\eta \nonumber \\
  &=\f{1}{kT}\Big(\pderiv{\Phi}{\eta}\Big)_{T}\,\Big(\pderiv{\mu_{e}}{\rho}\Big)_{T,Y_{e}}\,d\rho
  +\Big\{\,\Big(\pderiv{\Phi}{T}\Big)_{\eta} + \Big(\pderiv{\Phi}{\eta}\Big)_{T}\,\Big[\,\f{1}{kT}\Big(\pderiv{\mu_{e}}{T}\Big)_{\rho,Y_{e}}-\f{\eta}{T}\,\Big]\,\Big\}\,dT \nonumber \\
  &\hspace{12pt}
  +\f{1}{kT}\Big(\pderiv{\Phi}{\eta}\Big)_{T}\,\Big(\pderiv{\mu_{e}}{Y_{e}}\Big)_{\rho,T}\,dY_{e},
  \label{eq:variation_T_Eta}
\end{align}
where we used
\begin{align}
  d\eta &= \Big(\pderiv{\eta}{\rho}\Big)_{T,Y_{e}}\,d\rho+\Big(\pderiv{\eta}{T}\Big)_{\rho,Y_{e}}\,dT+\Big(\pderiv{\eta}{Y_{e}}\Big)_{\rho,T}\,dY_{e} \nonumber \\
  &=\f{1}{kT}\Big(\pderiv{\mu_{e}}{\rho}\Big)_{T,Y_{e}}\,d\rho 
  + \Big[\,\f{1}{kT}\Big(\pderiv{\mu_{e}}{T}\Big)_{\rho,Y_{e}}-\f{\eta}{T}\,\Big]\,dT + \f{1}{kT}\Big(\pderiv{\mu_{e}}{Y_{e}}\Big)_{\rho,T}\,dY_{e}.  
\end{align}

\begin{align}
  d\Phi
  &=\Big(\pderiv{\Phi}{\epsilon}\Big)_{Y_{e}}\,d\epsilon+\Big(\pderiv{\Phi}{Y_{e}}\Big)_{\epsilon}\,dY_{e} \nonumber \\
  &=\Big(\pderiv{\Phi}{\epsilon}\Big)_{Y_{e}}\,\Big(\pderiv{\epsilon}{\rho}\Big)_{T,Y_{e}}\,d\rho
  +\Big(\pderiv{\Phi}{\epsilon}\Big)_{Y_{e}}\,\Big(\pderiv{\epsilon}{T}\Big)_{\rho,Y_{e}}\,dT \nonumber \\
  &\hspace{12pt}
  +\Big[\,\Big(\pderiv{\Phi}{\epsilon}\Big)_{Y_{e}}\,\Big(\pderiv{\epsilon}{Y_{e}}\Big)_{\rho,T}+\Big(\pderiv{\Phi}{Y_{e}}\Big)_{\epsilon}\,\Big]\,dY_{e},
  \label{eq:variation_Epsilon_Ye}
\end{align}
where we used
\begin{equation}
  d\epsilon
  =\Big(\pderiv{\epsilon}{\rho}\Big)_{T,Y_{e}}\,d\rho+\Big(\pderiv{\epsilon}{T}\Big)_{\rho,Y_{e}}\,dT+\Big(\pderiv{\epsilon}{Y_{e}}\Big)_{\rho,T}\,dY_{e}
\end{equation}
Comparing Eq.~\eqref{eq:variation_T_Eta} and Eq.~\eqref{eq:variation_Epsilon_Ye}, we have
\begin{align}
  \Big(\pderiv{\Phi}{T}\Big)_{\eta} + \Big[\,\f{1}{kT}\Big(\pderiv{\mu_{e}}{T}\Big)_{\rho,Y_{e}}-\f{\eta}{T}\,\Big]\,\Big(\pderiv{\Phi}{\eta}\Big)_{T}
  &=\Big(\pderiv{\epsilon}{T}\Big)_{\rho,Y_{e}}\,\Big(\pderiv{\Phi}{\epsilon}\Big)_{Y_{e}}, \\
  \f{1}{kT}\Big(\pderiv{\mu_{e}}{Y_{e}}\Big)_{\rho,T}\,\Big(\pderiv{\Phi}{\eta}\Big)_{T}
  &=\Big(\pderiv{\epsilon}{Y_{e}}\Big)_{\rho,T}\,\Big(\pderiv{\Phi}{\epsilon}\Big)_{Y_{e}}+\Big(\pderiv{\Phi}{Y_{e}}\Big)_{\epsilon}
\end{align}
Solving for $(\partial\Phi/\partial\epsilon)_{Y_{e}}$ and $(\partial\Phi/\partial Y_{e})_{\epsilon}$ we obtain
\begin{align}
  \Big(\pderiv{\Phi}{\epsilon}\Big)_{Y_{e}}
  &=\Big[\,\f{1}{kT}\Big(\pderiv{\mu_{e}}{T}\Big)_{\rho,Y_{e}}-\f{\eta}{T}\,\Big]/\Big(\pderiv{\epsilon}{T}\Big)_{\rho,Y_{e}}\,\Big(\pderiv{\Phi}{\eta}\Big)_{T} + \Big(\pderiv{\epsilon}{T}\Big)_{\rho,Y_{e}}^{-1}\,\Big(\pderiv{\Phi}{T}\Big)_{\eta}, \\
  \Big(\pderiv{\Phi}{Y_{e}}\Big)_{\epsilon}
  &=\Big\{\,\f{1}{kT}\Big(\pderiv{\mu_{e}}{Y_{e}}\Big)_{\rho,T} - \Big(\pderiv{\epsilon}{Y_{e}}\Big)_{\rho,T}\,\Big[\,\f{1}{kT}\Big(\pderiv{\mu_{e}}{T}\Big)_{\rho,Y_{e}}-\f{\eta}{T}\,\Big]/\Big(\pderiv{\epsilon}{T}\Big)_{\rho,Y_{e}}\,\Big\}\,\Big(\pderiv{\Phi}{\eta}\Big)_{T} \nonumber \\
  &\hspace{12pt}
  - \Big\{\,\Big(\pderiv{\epsilon}{Y_{e}}\Big)_{\rho,T}/\Big(\pderiv{\epsilon}{T}\Big)_{\rho,Y_{e}}\,\Big\}\,\Big(\pderiv{\Phi}{T}\Big)_{\eta}.
\end{align}

\end{document}